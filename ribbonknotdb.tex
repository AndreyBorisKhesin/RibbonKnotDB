\documentclass[twoside]{article}
\usepackage{amsmath}
\usepackage{amssymb}
\usepackage{amsthm}
\usepackage{capt-of}
\usepackage{caption}
\captionsetup{labelformat=empty,labelsep=none}
\usepackage[strict]{changepage}
\usepackage{chngcntr}
\usepackage[americanvoltage, siunitx]{circuitikz}
\usepackage{color,colortbl}
\usepackage{fancyhdr}
\usepackage[T1]{fontenc}
\usepackage{gensymb}
\usepackage[margin=1in]{geometry}
\usepackage{graphicx}
\usepackage{import}
\usepackage{indentfirst}
\usepackage{mathptmx}
\usepackage{mathrsfs}
\usepackage{multicol}
\usepackage{multirow}
\usepackage{pgfplots}
\usepackage{pgfplotstable}
\usepgfplotslibrary{external}
\usepackage{siunitx}
\usepackage{tabu}
\usepackage{tikz}
\usetikzlibrary{positioning,matrix,shapes,chains,arrows}
\tikzexternalize[prefix=precompiled_figures/]
\usepackage{xspace}

% Indent
\setlength{\parindent}{0.3in}

\newcounter{papertheoremamount}
\newcommand\theorems[0]{\newtheorem{conjecture}[subsection]{Conjecture}
\newtheorem{corollary}[subsection]{Corollary}
\newtheorem{lemma}[subsection]{Lemma}\newtheorem{remark}[subsection]{Remark}
\newtheorem{theorem}[subsection]{Theorem}
\newtheorem{question}[subsection]{Question}}
\newcommand\subtheorems[0]{\stepcounter{papertheoremamount}
\newtheorem{conjecture}[subsubsection]{Conjecture}
\newtheorem{corollary}[subsubsection]{Corollary}
\newtheorem{lemma}[subsubsection]{Lemma}
\newtheorem{remark}[subsubsection]{Remark}
\newtheorem{theorem}[subsubsection]{Theorem}
\newtheorem{question}[subsubsection]{Question}}

% Title section
\pagestyle{fancy}
\thispagestyle{empty}
\renewcommand{\headrulewidth}{0pt}
\newcommand\papertitle[1]{{\centering\fontsize{20pt}{1em}\textsc{#1}\\\mbox{}\\}
\fancyhead[OC]{\fontsize{12pt}{12pt}\selectfont\textit{#1}}}
\newcounter{people}
\newcommand\paperauthortext[4]{{\centering\fontsize{12pt}{1em}\selectfont
\textsc{#1. #2}\\[-0.1em]{\fontsize{9pt}{1em}\selectfont\textit{\ifx&#3&
\vspace{-1em}\else#3\fi}}\\\mbox{}\\
\fancyhead[EC]{\fontsize{12pt}{12pt}\selectfont\textit{#4}}}}
\newcommand\paperauthor[3]{{\stepcounter{people}
\ifnum\value{people}=1
{\paperauthortext{#1}{#2}{#3}{#1. #2}
\global\def\auth{#2\xspace}}
\else\ifnum\value{people}=2
{\paperauthortext{#1}{#2}{#3}{\auth and #2}}
\else{\paperauthortext{#1}{#2}{#3}{\auth et al}}\fi\fi}}
\newcommand\paperdate[1]{{\centering\fontsize{9pt}{1em}\selectfont\text{
(Received #1)}\\[2em]}}

% Page header
\newcommand{\paperheader}[1]{\fancyhead[EC]{\fontsize{12pt}{12pt}\selectfont
\textit{#1}}}
\fancyhead[RO, EL]{\fontsize{12pt}{12pt}\selectfont\thepage}
\cfoot{}

\makeatletter
\newenvironment{paperadjustwidth}[2]{
  \begin{list}{}{
    \setlength\partopsep\z@
    \setlength\topsep\z@
    \setlength\listparindent\parindent
    \setlength\parsep\parskip
    \linespread{0}\selectfont
    \@ifmtarg{#1}{\setlength{\leftmargin}{\z@}}
                 {\setlength{\leftmargin}{#1}}
    \@ifmtarg{#2}{\setlength{\rightmargin}{\z@}}
                 {\setlength{\rightmargin}{#2}}
    }
    \item[]}{\end{list}}
\makeatother

% Abstract environment
\newenvironment{paperabstract}
{\begin{paperadjustwidth}{0.5in}{0.5in}\bgroup\fontsize{9pt}{1em}\selectfont
\hspace{0.5in}}
{\egroup\end{paperadjustwidth}}

% Report environment
\setlength\columnsep{0.5in}
\newenvironment{paper}
{\begin{multicols*}{2}\bgroup\fontsize{12pt}{13.8pt}\selectfont}
{\egroup\end{multicols*}}

%Sources
\newsavebox{\sourcebox}
\newcommand{\papersource}[1]{
\vspace{-2em}
\text{}\\*
\fontsize{9pt}{10.35}\selectfont
\noindent\renewcommand{\labelenumi}{}
\savebox{\sourcebox}{\parbox{3in}{\begin{enumerate}
%\vspace{-\baselineskip}
\setlength{\leftmargini}{-1ex}
\setlength{\leftmargin}{-1ex}
\setlength{\labelwidth}{0pt}
\setlength{\labelsep}{0pt}
\setlength{\listparindent}{0pt}
\item\textit{\hspace{-0.35in}#1}
\end{enumerate}}}
\usebox{\sourcebox}
}

%Section headers
\newcounter{papersectioncounter}
\newcounter{papersubsectioncounter}[papersectioncounter]
\newcommand\papersection[1]{\stepcounter{papersectioncounter}
\stepcounter{section}
\begin{center}\Roman{papersectioncounter} \textsc{#1}\end{center}}
\newcommand\papersubsection[1]{\stepcounter{papersubsectioncounter}
\addtocounter{subsection}{\thepapertheoremamount}
\setcounter{subsubsection}{0}
{\begin{center}
\Roman{section}.\Roman{papersubsectioncounter}
\textsc{#1}\\[0.5em]\end{center}}}

%equation
\newcounter{paperequationcounter}
\newcommand\paperequation[3]{{
\stepcounter{paperequationcounter}
\mbox{}\vspace{-0.75em}
\begin{equation*}
#1
\tag*{\fontsize{12pt}{1em}\selectfont
$\begin{array}{r}
\cr{\text{[\arabic{paperequationcounter}]}}
\cr{\fontsize{9pt}{1em}\selectfont\textit{\ifx\\#2\\~\else(\fi#2\ifx\\#2\\~
\else)\fi}}
\end{array}$}
\end{equation*}
}
\expandafter\edef\csname eq#3\endcsname{[\arabic{paperequationcounter}]\noexpand
\xspace}}

% Where
\newcommand{\paperwherevar}[3]{&$#1$ & #2 \ifx\\#3\\~\else($\smash{\text{\si{\fi
#3\ifx\\#3\\~\else}}}$)\fi\\}
\newenvironment{paperwhere}
{\bgroup\fontsize{9pt}{9pt}\selectfont Where:\vspace{2pt}\\\begin{tabular}
{rr@{ = }p{2.42in}}}
{\end{tabular}\egroup\vspace{5pt}}

% Tables
\definecolor{LineGray}{gray}{0.5}
\newtabulinestyle{outer=2.25pt LineGray}
\newtabulinestyle{inner=0.75pt LineGray}
\tabulinesep=1.5pt

\newcommand{\paperiline}[0]{\tabucline[inner]{-}}
\newcommand{\paperoline}[0]{\tabucline[outer]{-}}

% Index column type
\newcolumntype{I}{X[-5,c]}
% Column type with uncertainty
\newcolumntype{U}{@{}X[-5,r]@{$\pm$}X[-5,l]@{}}
% Column type without uncertainty
\newcolumntype{C}{@{}X[-5,c]@{}}

\newcounter{papertableindexcounter}
\newcommand{\papertableindexheader}[0]{\multirow{2}{*}{\textsc{Index}}}
\newcommand{\papertableindex}[0]{\stepcounter{papertableindexcounter}
\arabic{papertableindexcounter}}
\newcommand{\papertableuheadersymbol}[1]{&\multicolumn{2}{c|[inner]}{$#1$}}
\newcommand{\papertableuheadersymbole}[1]{&\multicolumn{2}{c|[outer]}{$#1$}}
\newcommand{\papertableuheaderunit}[1]{&\multicolumn{2}{c|[inner]}{(#1)}}
\newcommand{\papertableuheaderunite}[1]{&\multicolumn{2}{c|[outer]}{(#1)}}
\newcommand{\papertablecheadersymbol}[1]{&$#1$}
\newcommand{\papertablecheaderunit}[2]{&($\pm$#1 #2)}

% Value in table with uncertainty.
\newcommand{\papertableuval}[2]{& #1 & #2}
% Value in table without uncertainty.
\newcommand{\papertablecval}[1]{& #1}

\newenvironment{papertable}[1]
{\setcounter{papertableindexcounter}{0} 
\begin{tabu} to \linewidth {#1}}
{\end{tabu}\vspace{12pt}}

%Figure counter
\newcounter{paperfigurecounter}
\newcommand{\papercaption}[1]{
\vspace{-24pt}
\bgroup
\stepcounter{paperfigurecounter}
\captionof{figure}{\fontsize{9pt}{9pt}\selectfont
\hspace{0.3in}Fig \arabic{paperfigurecounter}. #1}\egroup}

\newcommand{\paperaxis}[9]
{title=#1,
axis x line = bottom,
xmin=#4,xmax=#6,
axis y line = left,
ymin=#5,ymax=#7,
height = 180pt,
grid=both,
x axis line style=-,
y axis line style=-,
x tick label style={
/pgf/number format/.cd,
fixed,
fixed zerofill,
precision=#8,
/tikz/.cd
},
y tick label style={
/pgf/number format/.cd,
fixed,
fixed zerofill,
precision=#9,
/tikz/.cd
}
}
\newcommand{\paperaxisxlabel}[2]{
xlabel=\fontsize{10pt}{1em}\selectfont#1$(#2)\rightarrow$}
\newcommand{\paperaxisylabel}[2]{
ylabel=\fontsize{10pt}{1em}\selectfont#1$(#2)\rightarrow$}
\newcommand{\papergraphoutline}[4]{
\addplot [mark=none,line width=0.75pt] coordinates {
(#1,#2)
(#1,#4)
(#3,#4)
(#3,#2)
(#1,#2)
};}

\newenvironment{papergraph}{
\begin{tikzpicture}
\begin{axis}
}
{\end{axis}
\end{tikzpicture}}

\newcommand{\abs}[1]{\left\lvert#1\right\rvert}
\newcommand{\oo}[0]{\infty}
\newcommand{\sigmaSum}[3]{\sum\limits_{#1}^{#2} #3}
\newcommand{\limto}[3]{\lim\limits_{#1\rightarrow#2}#3}
\renewcommand{\d}[0]{\mathrm{d}}
\newcommand{\cross}[0]{\times}
\newcommand{\lp}{\left(}
\newcommand{\rp}{\right)}
\newcommand\pars[1]{\lp#1\rp}
\newcommand\sqbrack[1]{\left[#1\right]}
\newcommand\R{\mathbb{R}}
\newcommand\di{\partial}
\newcommand\x{\times}
\newcommand\del{\nabla}

\theorems
\begin{document}
\papertitle{A Database of Ribbon Knots in Tangle Form}
\paperauthor{D}{Bar-Natan}{}
\paperauthor{A}{Khesin}{}
\paperdate{TODO DATE}
\begin{paperabstract}
TODO ABSTRACT
\end{paperabstract}
\begin{paper}
\papersection{Introduction}

A ribbon knot is a knot that contains only ribbon singularities.
A ribbon singularity occurs when the disk that is bounded by the knot only
intersects itself in pairs, so when the boundary of the disk passes through
said disk, there will be another intersection near the first with the opposite
sign, meaning the boundary crosses the disk from the other side as the first.
A ribbon presentation for a knot depicts the knot as the image of a series
of disks, each connected to the next by an arc, the ribbon, which is split into
two strands after applying the map.

A symmetric union is a knot that is symmetric about a central axis, save the
crossings that lie on the axis itself.
It is well known that any symmetric union is a ribbon knot.
However, at this time of this writing it is not yet known whether every ribbon
knot can be expressed as a symmetric union.
For all 21 knots with 10 or fewer crossings it has been shown that they can be
expressed as symmetric unions (\cite{oneknot} and \cite{manyknots}).
\papersection{Tangles}

A tangle is a collection of joined crossings with a certain number of ``loose
ends'' sticking out.
A tangle with $n$ strands will have $2n$ such ends sticking out.
For convenience, a tangle is drawn similarly to a pure braid, with $n$ ends
along the top and $n$ ends along the bottom where the order of the strands along
the top is the same as along the bottom.
In other words, if one were to follow each strand in a tangle, they would end up
exactly below the point at which they started.
Let the tops of the strands be denoted by $1_t$, $2_t$, ..., $n_t$, and the
bottoms  by $1_b$, $2_b$, ..., $n_b$.

\svg{diagrams/tangle}
\papercaption{Tangle}{In a tangle, the ends do not have to be drawn along the
top and bottom of the tangle.
This is merely done for convenience.
In practice, the $2n$ ends could be located anywhere on the perimeter.
Here the central $R$ represents a knot that shows that the strands are tangled,
but link up with their corresponding ends the way they do in a pure braid.}

A closure of a tangle is a manner of stitching the ends of the tangle together
to reduce the resulting number of strands.
For a tangle with $2n$ strands, let the top closure of the tangle be defined as
the closure that stitches together the pairs $(1_t$, $2_t)$, $(3_t$, $4_t)$,
\dots, $((2n-1)_t$, $(2n)_t)$ to create $n$ strands.

\svg{diagrams/top}
\papercaption{Top}{This shows how a tangle with $2n$ strands can be top closed.
This leaves every bottom end open.
Now, this tangle has only $n$ strands instead of $2n$.
Notice that only 4 of the $n$ closures are shown here.
Here the central $R$ represents a knot that shows that the strands are tangled,
but link up with their corresponding ends the way they do in a pure braid.}

For a tangle with $2n$ strands, let the full closure of the tangle be defined as
the closure that stitches together the pairs $(1_b$, $2_b)$, $(2_t$, $3_t)$,
$(3_b$, $4_b)$, $...$, $((2n-2)_t$, $(2n-1)_t)$, $((2n-1)_b$, $(2n)_b)$ to
create 1 strand.

\svg{diagrams/full}
\papercaption{Full}{This shows how a tangle with $2n$ strands can be full
closed.
It is important to note that this reproduces the top closure, but along the
bottom, and closed the top similarly, but offset by 1 strand.
This leaves ends $1_t$ and $(2n)_t$ open.
Now, this tangle has only 1 strand instead of $2n$.
Notice that only 8 of the $2n-1$ closures are shown here.
The break in the central top closure represents the presence of more closed
strands.
Here the central $R$ represents a knot that shows that the strands are tangled,
but link up with their corresponding ends the way they do in a pure braid.}

Note that in both scenarios, one is effectively closing the strands along the
bottom in the same manner.
In the top closure, the strands create an untangle which can be turned into the
unlink by closing the strands along the bottom as it is done in the
full closure.

\begin{papertheorem}{ribbon}
A knot $K$ is ribbon$\iff\exists$ a tangle $T$ with $2n$ strands where the
top closure of $T$ creates the trivial tangle with $n$ strands, while
full closure of $T$ creates a tangle with 1 strand, identical to the knot $K$
(obtained by stitching $(1_t$, $(2n)_t)$).
\end{papertheorem}
\begin{proof}
If the top closure of a tangle $T$ results in $n$ trivial strands that can be
untangled, that means among the crossings that make up the top closure of $T$,
there are only ribbon singularities, as any clasp singularities would not make
the top closure of $T$ trivial.
Now those $n$ strands can be closed along the bottom to create $n$ unknots.
Then each unknot is connected to the next using a single strand.
That strand may pass through the plane of the original knot, but if it were to
be unzipped into strands, it would result in only ribbon singularities.
This has effectively connected every adjacent pair of links as well as the first
and last links together, creating the original knot by the assumption.
Since the resultant knot only contains ribbon singularities, $K$ is ribbon.

Conversely, consider a tangle $T$ with $n$ strands and $2n$ ends along the top
numbered $1_t$ to $(2n)_t$.
\begin{paperlemma}{Tangles}
If a tangle $T$ with $n$ strands with $2n$ ends along the top can be top closed
(with $n$ stitchings) to create the unlink consisting of $n$ unknots, then
$\exists$ a tangle $T'$ consisting of $2n$ strands with $2n$ ends along the top
and $2n$ ends along the bottom such that applying the stitchings$(1_b$, $2_b)$,
$(2_t$, $3_t)$, $(3_b$, $4_b)$, $...$, $((2n-2)_t$, $(2n-1)_t)$, $((2n-1)_b$,
$(2n)_b)$ (just the bottom part of the full closure) results in $T$.

\svg{diagrams/lemma}
\papercaption{Lemma}{The top equality shows that the tangle $T$ can be top
closed to form the unlink $U$.
The implication states that there must exist a tangle $T'$ that can be closed
along the bottom as shown to result in $T$.
The break in the central top closure represents the presence of more closed
strands.}
\end{paperlemma}
\begin{proof}[Proof of Lemma]
Once $T$ is closed, a continuous deformation of three dimensional space turns it
into the unlink.
Here, the $n$ segments from the unlink are extended to points outside the
tangle, making sure that they do not intersect.
This can be done trivially as $T$ is untangled.
If the earlier spatial deformation were to be reversed, the result will be a
tangle with $2n$ ends along both the top and the bottom, where the bottom ends
extend to be stitched pairwise as described earlier and the tangle can be top
closed to produce the untangle.
This tangle is therefore $T'$.
TODO DIAGRAM
\end{proof}
For any knot in its ribbon presentation, the lower half of each component can be
enclosed in one large tangle.
The knot can be deformed until the ribbon connects the tops of the knot's
components, making sure the ends of the ribbons are above the upper boundary of
the tangle.

\svg{diagrams/lowered}
\papercaption{Lowered}{The intersections of the ribbon and the components of the
ribbon knot can be lowered into the enclosed section labeled $R$.
The components of the knots are depicted exactly as they are while the
intersections of the ribbons and the components are not shown.
The break in the central top closure represents the presence of more closed
strands.}

It is important to note that everything above the upper boundary of the tangle
is exactly as described while the ribbons inside the tangle can have any number
of crossings.
Since any space that is in the tangle but above the ribbon is connected to the
outside of the knot, any strands that pass through it can be moved up and around
the knot to end up fully inside the tangle.
This means that any ribbon knot with $n$ components can be transformed into a
tangle with $2n$ ends along the top that has been top closed with $n-1$ ribbon
bridges between each of the closing loops.

\svg{diagrams/twisted}
\papercaption{Twisted}{This diagram is very similar to \figLowered but the knot
it represents is very different.
Here, every ribbon, originally twisted with the components of the ribbon knot,
was dragged to the position it occupies here, dragging the strands of the
components with it.
Since every area outside of the enclosed section labeled $R$, whether it be
above the knot or below the ribbons, is connected to $R$, then every strand that
got dragged with the ribbons can be moved to within $R$ once again.
This results in the components inside of $R$ being knotted, whereas before, they
were not.
The break in the central top closure represents the presence of more closed
strands.}

Here the knot in the tangle is completely different and all the crossings are in
the components of the knot, as opposed to the ribbon.
This is equivalent to applying just the top part of the full closure to the
tangle and then applying the closure $(1_t$, $(2n)_t)$.

\svg{diagrams/complete}
\papercaption{Complete}{In this diagram, nothing within $R$ changed from
\figTwisted.
Everything above $R$ was smoothly deformed to illustrate that the resulting
strands above $R$ are simply a top closure and the connection $(1_t$,
$(2n)_t)$.}

Note that since any ribbon knot is a series of unzipped strands connecting
subsequent components of the unlink, then the removal of these strands will
result in the unlink.
Similarly, if a ribbon knot were to be deformed in the above manner, then
removing the ribbons would also result in the unlink.
Note that in any situation, removing a ribbon is equivalent to unzipping
it, cutting both strands of the ribbon, then stitching each strand to the one it
was unzipped from.

\svg{diagrams/cut}
\papercaption{Cut}{If a ribbon connecting two strands were to be entirely
removed, this would be equivalent to severing both strands of the ribbon and
reattaching each strand to its neighbour, but with an offset of 1, thus
attaching each strand to the neighbour that it was not attached to before.}

Removing the ribbons from the deformed ribbon knot would therefore be identical
to applying the top closure to its tangle.
This means that the tangle that makes up the bulk of the knot satisfies the
conditions for $T$ in \lemTangles, thus there exists a similar tangle T', that
satisfies the conditions of the tangle described in the lemma.
\end{proof}
\papersection{Symmetric Unions}

Here is a list of all 21 ribbon knots with 10 crossings or fewer.
Each was originally a symmetric union provided by \cite{oneknot} and
\cite{manyknots} and was deformed to appear as the full closure of a tangle with
the extra stitching of $4_t$ and $1_t$.
\end{paper}

\setlength{\tabcolsep}{12pt}
\fontsize{12pt}{1em}\selectfont
\begin{tabular}{cccc}
\scalebox{0.17}{\svg{diagrams/6_1}}&\scalebox{0.17}{\svg{diagrams/8_8}}&
\scalebox{0.17}{\svg{diagrams/8_9}}&\scalebox{0.17}{\svg{diagrams/8_20}}\\
$6_1$&$8_8$&$8_9$&$8_{20}$\\
&&&\\
\scalebox{0.17}{\svg{diagrams/9_27}}&\scalebox{0.17}{\svg{diagrams/9_41}}&
\scalebox{0.17}{\svg{diagrams/9_46}}&\scalebox{0.17}{\svg{diagrams/10_3}}\\
$9_{27}$&$9_{41}$&$9_{46}$&$10_3$\\
&&&\\
\scalebox{0.17}{\svg{diagrams/10_22}}&\scalebox{0.17}{\svg{diagrams/10_35}}&
\scalebox{0.17}{\svg{diagrams/10_42}}&\scalebox{0.17}{\svg{diagrams/10_48}}\\
$10_{22}$&$10_{35}$&$10_{42}$&$10_{48}$\\
&&&\\
\scalebox{0.17}{\svg{diagrams/10_75}}&\scalebox{0.17}{\svg{diagrams/10_87}}&
\scalebox{0.17}{\svg{diagrams/10_99}}&\scalebox{0.17}{\svg{diagrams/10_123}}\\
$10_{75}$&$10_{87}$&$10_{99}$&$10_{123}$\\
&&&\\
\scalebox{0.17}{\svg{diagrams/10_129}}&\scalebox{0.17}{\svg{diagrams/10_137}}&
\scalebox{0.17}{\svg{diagrams/10_140}}&\scalebox{0.17}{\svg{diagrams/10_153}}\\
$10_{129}$&$10_{137}$&$10_{140}$&$10_{153}$\\
&&&\\
\scalebox{0.17}{\svg{diagrams/10_155}}&&&\\
$10_{155}$&&&
\end{tabular}

\begin{paper}
The locations of the cuts that result in the ends $1_b$ and $2_b$ as well as
$3_b$ and $4_b$ are, in general, not very interesting.
It is merely a location on the knot that can be kept on the outside while
untangling the top closure of the tangle.
Note that in the symmetric union, the locations of the two important cuts can
always be obtained by deforming the knot (but preserving its symmetry) until one
of the bridges appears above any crossings on the axis of symmetry.
Then the knot is cut just above this bridge along the first strands that cross
it, moving out from the middle of the bridge.

\newsavebox{\mirrorR}
\sbox{\mirrorR}{\reflectbox{\fontsize{9pt}{1em}\selectfont$R$}}

\svg{diagrams/question}
\papercaption{Question}{Here, $R$ represents a knot that is connected to its
mirror image, \usebox{\mirrorR}, by a series of crossings, $X$.
$U$ represents the unlink.
The symmetric union has two bridges that cross the axis, connecting $R$ to
\usebox{\mirrorR}.
As the theorem states, there exists a way to cut the symmetric union in two four
places and then stitch them as described to create the untangle.
Is it true that one such location of the two cuts that end up along the top of
the tangle can be along the strands that first cross the upper bridge?}

\begin{paperquestion}{Cuts}
For any ribbon knot in its symmetric union presentation, can the location of the
cuts between $(1_t$, $4_t)$ and $(2_t$, $3_t)$ be generalized to be along the
two strands that are first to cross one of the 2 bridges of the symmetric union?
\end{paperquestion}
\papersection{Database}

Here is a database of the planar diagram crossing information of the 21 ribbon
knots.
They are $6_1$, $8_8$, $8_9$, $8_{20}$, $9_{27}$, $9_{41}$, $9_{46}$, $10_3$,
$10_{22}$, $10_{35}$, $10_{42}$, $10_{48}$, $10_{75}$, $10_{87}$, $10_{99}$,
$10_{123}$, $10_{129}$, $10_{137}$, $10_{140}$, $10_{153}$, and $10_{155}$.
This notation consists of a list of crossings of the form $X_{a,b,c,d}$ which
labels the four strands in a crossing from the lower incoming strand and
continues counterclockwise.

~~~~~~~~~~~\scalebox{0.5}{\svg{diagrams/crossing}}
\vspace{12pt}
\papercaption{Crossing}{This shows a typical crossing.
The numbering stars from the lower incoming segment and proceeds
counterclockwise.
Thus, this crossing would be labeled $X_{a,b,c,d}$.}

The knots are encoded as they appear in TODO FIGURE NUMBER.
The tangle information consists of the list of segments of the knot (before any
stitching between crossings is done) of each strand of the tangle.
A strand in a crossing will carry the number of the knot segment going into it.
Here is knot $6_1$ as an example.

\svg{diagrams/example}
\papercaption{Example}{This depicts all the strands of knot $6_1$ as they would
be numbered.
The numbering always starts from strand 1 and proceeds along the knot.
The crossing information of this knot would be $X_{2,8,3,7}$ $X_{5,14,6,15}$
$X_{8,20,9,19}$ $X_{3,10,4,11}$ $X_{11,6,12,7}$ $X_{15,4,16,5}$
$X_{17,16,18,17}$ $X_{18,10,19,9}$ $X_{20,2,21,1}$ $X_{21,12,22,13}$
$X_{22,14,1,13}$.
Meanwhile, the tangle information of this knot would be \{\{1, 2, 3, 4, 5, 6, 7,
8\}, \{12, 11, 10, 9\}, \{13, 14, 15, 16\}, \{22, 21, 20, 19, 18, 17\}\}.}

The full database may be accessed at the following URL.
TODO URL
\papersection{Acknowledgments}
\papersection{References}
\begin{thebibliography}{}
\bibitem{oneknot}
Lamm, Christoph.
\textit{Symmetric union presentations for 2-bridge ribbon knots.}
\texttt{arXiv:math/0602395}
\bibitem{manyknots}
Lamm, Christoph.
\textit{Symmetric unions and ribbon knots.}
Osaka J. Math. 37 (2000), no. 3, 537--550
\end{thebibliography}
\end{paper}
\end{document}
