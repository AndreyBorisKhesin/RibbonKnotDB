\documentclass[twoside]{article}
\usepackage{amsmath}
\usepackage{amssymb}
\usepackage{amsthm}
\usepackage{capt-of}
\usepackage{caption}
\captionsetup{labelformat=empty,labelsep=none}
\usepackage[strict]{changepage}
\usepackage{chngcntr}
\usepackage[americanvoltage, siunitx]{circuitikz}
\usepackage{color,colortbl}
\usepackage{fancyhdr}
\usepackage[T1]{fontenc}
\usepackage{gensymb}
\usepackage[margin=1in]{geometry}
\usepackage{graphicx}
\usepackage{import}
\usepackage{indentfirst}
\usepackage{mathptmx}
\usepackage{mathrsfs}
\usepackage{multicol}
\usepackage{multirow}
\usepackage{pgfplots}
\usepackage{pgfplotstable}
\usepgfplotslibrary{external}
\usepackage{siunitx}
\usepackage{tabu}
\usepackage{tikz}
\usetikzlibrary{positioning,matrix,shapes,chains,arrows}
\tikzexternalize[prefix=precompiled_figures/]
\usepackage{xspace}

% Indent
\setlength{\parindent}{0.3in}

\newcounter{papertheoremamount}
\newcommand\theorems[0]{\newtheorem{conjecture}[subsection]{Conjecture}
\newtheorem{corollary}[subsection]{Corollary}
\newtheorem{lemma}[subsection]{Lemma}\newtheorem{remark}[subsection]{Remark}
\newtheorem{theorem}[subsection]{Theorem}
\newtheorem{question}[subsection]{Question}}
\newcommand\subtheorems[0]{\stepcounter{papertheoremamount}
\newtheorem{conjecture}[subsubsection]{Conjecture}
\newtheorem{corollary}[subsubsection]{Corollary}
\newtheorem{lemma}[subsubsection]{Lemma}
\newtheorem{remark}[subsubsection]{Remark}
\newtheorem{theorem}[subsubsection]{Theorem}
\newtheorem{question}[subsubsection]{Question}}

% Title section
\pagestyle{fancy}
\thispagestyle{empty}
\renewcommand{\headrulewidth}{0pt}
\newcommand\papertitle[1]{{\centering\fontsize{20pt}{1em}\textsc{#1}\\\mbox{}\\}
\fancyhead[OC]{\fontsize{12pt}{12pt}\selectfont\textit{#1}}}
\newcounter{people}
\newcommand\paperauthortext[4]{{\centering\fontsize{12pt}{1em}\selectfont
\textsc{#1. #2}\\[-0.1em]{\fontsize{9pt}{1em}\selectfont\textit{\ifx&#3&
\vspace{-1em}\else#3\fi}}\\\mbox{}\\
\fancyhead[EC]{\fontsize{12pt}{12pt}\selectfont\textit{#4}}}}
\newcommand\paperauthor[3]{{\stepcounter{people}
\ifnum\value{people}=1
{\paperauthortext{#1}{#2}{#3}{#1. #2}
\global\def\auth{#2\xspace}}
\else\ifnum\value{people}=2
{\paperauthortext{#1}{#2}{#3}{\auth and #2}}
\else{\paperauthortext{#1}{#2}{#3}{\auth et al}}\fi\fi}}
\newcommand\paperdate[1]{{\centering\fontsize{9pt}{1em}\selectfont\text{
(Received #1)}\\[2em]}}

% Page header
\newcommand{\paperheader}[1]{\fancyhead[EC]{\fontsize{12pt}{12pt}\selectfont
\textit{#1}}}
\fancyhead[RO, EL]{\fontsize{12pt}{12pt}\selectfont\thepage}
\cfoot{}

\makeatletter
\newenvironment{paperadjustwidth}[2]{
  \begin{list}{}{
    \setlength\partopsep\z@
    \setlength\topsep\z@
    \setlength\listparindent\parindent
    \setlength\parsep\parskip
    \linespread{0}\selectfont
    \@ifmtarg{#1}{\setlength{\leftmargin}{\z@}}
                 {\setlength{\leftmargin}{#1}}
    \@ifmtarg{#2}{\setlength{\rightmargin}{\z@}}
                 {\setlength{\rightmargin}{#2}}
    }
    \item[]}{\end{list}}
\makeatother

% Abstract environment
\newenvironment{paperabstract}
{\begin{paperadjustwidth}{0.5in}{0.5in}\bgroup\fontsize{9pt}{1em}\selectfont
\hspace{0.5in}}
{\egroup\end{paperadjustwidth}}

% Report environment
\setlength\columnsep{0.5in}
\newenvironment{paper}
{\begin{multicols*}{2}\bgroup\fontsize{12pt}{13.8pt}\selectfont}
{\egroup\end{multicols*}}

%Sources
\newsavebox{\sourcebox}
\newcommand{\papersource}[1]{
\vspace{-2em}
\text{}\\*
\fontsize{9pt}{10.35}\selectfont
\noindent\renewcommand{\labelenumi}{}
\savebox{\sourcebox}{\parbox{3in}{\begin{enumerate}
%\vspace{-\baselineskip}
\setlength{\leftmargini}{-1ex}
\setlength{\leftmargin}{-1ex}
\setlength{\labelwidth}{0pt}
\setlength{\labelsep}{0pt}
\setlength{\listparindent}{0pt}
\item\textit{\hspace{-0.35in}#1}
\end{enumerate}}}
\usebox{\sourcebox}
}

%Section headers
\newcounter{papersectioncounter}
\newcounter{papersubsectioncounter}[papersectioncounter]
\newcommand\papersection[1]{\stepcounter{papersectioncounter}
\stepcounter{section}
\begin{center}\Roman{papersectioncounter} \textsc{#1}\end{center}}
\newcommand\papersubsection[1]{\stepcounter{papersubsectioncounter}
\addtocounter{subsection}{\thepapertheoremamount}
\setcounter{subsubsection}{0}
{\begin{center}
\Roman{section}.\Roman{papersubsectioncounter}
\textsc{#1}\\[0.5em]\end{center}}}

%equation
\newcounter{paperequationcounter}
\newcommand\paperequation[3]{{
\stepcounter{paperequationcounter}
\mbox{}\vspace{-0.75em}
\begin{equation*}
#1
\tag*{\fontsize{12pt}{1em}\selectfont
$\begin{array}{r}
\cr{\text{[\arabic{paperequationcounter}]}}
\cr{\fontsize{9pt}{1em}\selectfont\textit{\ifx\\#2\\~\else(\fi#2\ifx\\#2\\~
\else)\fi}}
\end{array}$}
\end{equation*}
}
\expandafter\edef\csname eq#3\endcsname{[\arabic{paperequationcounter}]\noexpand
\xspace}}

% Where
\newcommand{\paperwherevar}[3]{&$#1$ & #2 \ifx\\#3\\~\else($\smash{\text{\si{\fi
#3\ifx\\#3\\~\else}}}$)\fi\\}
\newenvironment{paperwhere}
{\bgroup\fontsize{9pt}{9pt}\selectfont Where:\vspace{2pt}\\\begin{tabular}
{rr@{ = }p{2.42in}}}
{\end{tabular}\egroup\vspace{5pt}}

% Tables
\definecolor{LineGray}{gray}{0.5}
\newtabulinestyle{outer=2.25pt LineGray}
\newtabulinestyle{inner=0.75pt LineGray}
\tabulinesep=1.5pt

\newcommand{\paperiline}[0]{\tabucline[inner]{-}}
\newcommand{\paperoline}[0]{\tabucline[outer]{-}}

% Index column type
\newcolumntype{I}{X[-5,c]}
% Column type with uncertainty
\newcolumntype{U}{@{}X[-5,r]@{$\pm$}X[-5,l]@{}}
% Column type without uncertainty
\newcolumntype{C}{@{}X[-5,c]@{}}

\newcounter{papertableindexcounter}
\newcommand{\papertableindexheader}[0]{\multirow{2}{*}{\textsc{Index}}}
\newcommand{\papertableindex}[0]{\stepcounter{papertableindexcounter}
\arabic{papertableindexcounter}}
\newcommand{\papertableuheadersymbol}[1]{&\multicolumn{2}{c|[inner]}{$#1$}}
\newcommand{\papertableuheadersymbole}[1]{&\multicolumn{2}{c|[outer]}{$#1$}}
\newcommand{\papertableuheaderunit}[1]{&\multicolumn{2}{c|[inner]}{(#1)}}
\newcommand{\papertableuheaderunite}[1]{&\multicolumn{2}{c|[outer]}{(#1)}}
\newcommand{\papertablecheadersymbol}[1]{&$#1$}
\newcommand{\papertablecheaderunit}[2]{&($\pm$#1 #2)}

% Value in table with uncertainty.
\newcommand{\papertableuval}[2]{& #1 & #2}
% Value in table without uncertainty.
\newcommand{\papertablecval}[1]{& #1}

\newenvironment{papertable}[1]
{\setcounter{papertableindexcounter}{0} 
\begin{tabu} to \linewidth {#1}}
{\end{tabu}\vspace{12pt}}

%Figure counter
\newcounter{paperfigurecounter}
\newcommand{\papercaption}[1]{
\vspace{-24pt}
\bgroup
\stepcounter{paperfigurecounter}
\captionof{figure}{\fontsize{9pt}{9pt}\selectfont
\hspace{0.3in}Fig \arabic{paperfigurecounter}. #1}\egroup}

\newcommand{\paperaxis}[9]
{title=#1,
axis x line = bottom,
xmin=#4,xmax=#6,
axis y line = left,
ymin=#5,ymax=#7,
height = 180pt,
grid=both,
x axis line style=-,
y axis line style=-,
x tick label style={
/pgf/number format/.cd,
fixed,
fixed zerofill,
precision=#8,
/tikz/.cd
},
y tick label style={
/pgf/number format/.cd,
fixed,
fixed zerofill,
precision=#9,
/tikz/.cd
}
}
\newcommand{\paperaxisxlabel}[2]{
xlabel=\fontsize{10pt}{1em}\selectfont#1$(#2)\rightarrow$}
\newcommand{\paperaxisylabel}[2]{
ylabel=\fontsize{10pt}{1em}\selectfont#1$(#2)\rightarrow$}
\newcommand{\papergraphoutline}[4]{
\addplot [mark=none,line width=0.75pt] coordinates {
(#1,#2)
(#1,#4)
(#3,#4)
(#3,#2)
(#1,#2)
};}

\newenvironment{papergraph}{
\begin{tikzpicture}
\begin{axis}
}
{\end{axis}
\end{tikzpicture}}

\newcommand{\abs}[1]{\left\lvert#1\right\rvert}
\newcommand{\oo}[0]{\infty}
\newcommand{\sigmaSum}[3]{\sum\limits_{#1}^{#2} #3}
\newcommand{\limto}[3]{\lim\limits_{#1\rightarrow#2}#3}
\renewcommand{\d}[0]{\mathrm{d}}
\newcommand{\cross}[0]{\times}
\newcommand{\lp}{\left(}
\newcommand{\rp}{\right)}
\newcommand\pars[1]{\lp#1\rp}
\newcommand\sqbrack[1]{\left[#1\right]}
\newcommand\R{\mathbb{R}}
\newcommand\di{\partial}
\newcommand\x{\times}
\newcommand\del{\nabla}

\begin{document}
\papertitle{TODO TITLE}
\paperauthor{D}{Bar-Natan}{TODO POSITION?}
\paperauthor{A}{Khesin}{TODO POSITION?}
\paperdate{TODO DATE}
\begin{paperabstract}
TODO ABSTRACT
\end{paperabstract}
\begin{paper}
\theorems
TODO ITALICIZE DEFINITIONS?
TODO DEFINITIONS?
TODO THE AUTHORS INSTEAD OF WE?
\papersection{Introduction}
A ribbon knot is a knot that contains only ribbon singularities.
A ribbon singularity occurs when the disk that is bounded by the knot only
intersects itself in pairs, e.g. when the boundary of the disk passes through
said disk, there will be another intersection ``nearby'' with the opposite sign.
A ``ribbon presentation'' for a knot depicts the knot as the image of a series
of disks, each connected to the next by an arc, the ribbon, which is ``split''
into two strands after applying the map.

A symmetric union is a knot that is symmetric about a central axis, save the
crossings that lie on the axis itself.
It is well-known that any symmetric union is a ribbon knot.
However, at this time of this writing TODO SOURCE it is not yet known whether
every ribbon knot can be expressed as a symmetric union.
For all 21 knots with 10 or fewer crossings it has been shown that they can be
expressed as symmetric unions TODO SOURCE.
\papersection{Tangles}
A tangle is a collection of joined crossings with a certain number of ``loose
ends'' sticking out.
A tangle with $n$ strands will have $2n$ such ends sticking out.
For convenience, a tangle is drawn as a braid, with $n$ ends along the top and
$n$ ends along the bottom where the order of the strands along the top is the
same as along the bottom.
In other words, if one were to follow each strand in a tangle, they would end up
exactly below the point at which they started.
Let the tops of the strands be denoted by $1_t,~2_t$, ..., $n_t$, and the
bottoms  by $1_b,~2_b$, ..., $n_b$.
TODO DIAGRAM

A closure of a tangle is a manner of stitching the ends of the tangle together
to reduce the resulting number of strands.
For a tangle with $2n$ strands, we define the top-closure of the tangle as the
closure that stitches together the pairs $(1_t,~2_t),~(3_t,~4_t)$, ...,
$((2n-1)_t,~(2n)_t)$ to create $n$ strands.
For a tangle with $2n$ strands, we define the full-closure of the tangle as the
closure that stitches together the pairs $(1_b,~2_b),~(2_t,~3_t),~(3_b$,
$4_b)$, ..., $((2n-2)_t,~(2n-1)_t),~((2n-1)_b,~(2n)_b)$ to create 1
strand.
TODO DIAGRAM

We point out that in both scenarios, one is effectively closing the strands
along the bottom in the same manner.
In the top-closure, the strands create an untangle which can be turned into the
unlink by closing the strands along the bottom as it is done in the
full-closure.

\begin{theorem}
A knot $K$ is ribbon$\iff\exists$ a tangle $T$ with $2n$ strands where the
top-closure of $T$ creates the trivial tangle with $n$ strands, while
full-closure of $T$ creates a tangle with 1 strand, identical to the knot $K$
(obtained by stitching $(1_t,~(2n)_t)$).
\end{theorem}
\begin{proof}
$(\impliedby)$
If the top-closure of a tangle $T$ results in $n$ trivial strands that can be
untangled, that means among the top-closure of $T$, there are only ribbon
singularities, as any clasp singularities would not make the top-closure of $T$
trivial.
We can now consider closing those $n$ strands along the bottom to create $n$
unknots.
Then each unknot is connected to the next using a single strand.
That strand may pass through the plane of the original knot, but if it were to
be unzipped into strands, it would result in only ribbon singularities.
This has effectively connected every adjacent pair of links as well as the first
and last links together, creating the original knot by our assumption.
Since the resultant knot only contains ribbon singularities, $K$ is ribbon.
\end{proof}
\begin{proof}
$(\implies)$
Consider a tangle $T$ with $n$ strands and $2n$ ends numbered $1_t$ to $(2n)_t$.

\end{proof}
\papersection{Symmetric Unions}
We now provide a list of all 21 ribbon knots with 10 crossings or fewer.
Each was originally a symmetric union provided by TODO SOURCE and was deformed
to appear as the full-closure of a tangle (with the extra stitching of $4_t$ and
$1_t$).
TODO DIAGRAM

The locations of the cuts that result in the ends $1_b$ and $2_b$ as well as
$3_b$ and $4_b$ are, in general, not very interesting.
It is merely a location on the knot that can be kept on the outside while
untangling the top-closure of the tangle.
We do note that in the symmetric union, the locations of the two important cuts
can always be obtained by deforming the knot (but preserving its symmetry) until
one of the bridges appears above any crossings on the axis of symmetry.
Then the knot is cut just above this bridge along the first strands that cross
it, moving out from the middle of the bridge.
TODO DIAGRAM
TODO QUESTION
\papersection{Database}
Here we attempt to publish a database of the planar diagram crossing information
of the 21 knots.
They are $6_1,~8_8,~8_9,~8_{20},~9_{27},~9_{41},~9_{46},~10_3$,
$10_{22},~10_{35},~10_{42},~10_{48},~10_{75},~10_{87},~10_{99}$,
$10_{123},~10_{129},~10_{137},~10_{140},~10_{153}$, and $10_{155}$.
This notation consists of a list of crossings of the form $X_{a,b,c,d}$ which
labels the four strands in a crossing from the lower incoming string and
continues counterclockwise.
TODO DIAGRAM

The knots are encoded as they appear in TODO FIGURE NUMBER.
The tangle information consists of the list of segments of the knot (before any
stitching between crossings is done) of each strand of the tangle.
A strand in a crossing will carry the number of the knot segment coming into it.
Here we present TODO KNOT as an example.
TODO DIAGRAM

The full database may be accessed at the following URL.
TODO URL
\papersection{A Correction}
We would like to use this opportunity to point out some slight errors in TODO
BOOK TITLE.
In Appendix TODO APPENDIX AND PAGE NUMBER, the ribbon presentations of the knots
TODO LIST KNOTS are incorrect, on account of the Jones polynomial of the
presentations not matching that of the intended knots.
Interestingly, all 4 knots evaluate to have the correct Alexander polynomial.
For this reason, we suspect the mistake was caused by one or more $\Delta\Delta$
moves while deforming the knot into the intended form.
\papersection{Acknowledgments}
\papersection{References}
\end{paper}
\end{document}
