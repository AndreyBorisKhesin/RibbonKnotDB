\documentclass[twoside]{article}
\usepackage{amsmath}
\usepackage{amssymb}
\usepackage{fancyhdr}
\usepackage[margin=1in]{geometry}
\usepackage{import}
\usepackage{multicol}
\usepackage[strict]{changepage}
\usepackage{color,colortbl}
\usepackage{tabu}
\usepackage{multirow}
\usepackage{capt-of}
\usepackage{caption}
\usepackage{tikz}
\usetikzlibrary{positioning,matrix,shapes,chains,arrows}
\usepackage{pgfplots}
\usepackage{pgfplotstable}
\captionsetup{labelformat=empty,labelsep=none}
\usepackage{graphicx}
\usepackage{gensymb}
\usepackage{siunitx}
\usepackage[americanvoltage, siunitx]{circuitikz}
\usepackage{mathrsfs}
\usepackage{xspace}

\usepgfplotslibrary{external}
\tikzexternalize[prefix=precompiled_figures/]

\usepackage[T1]{fontenc}
\usepackage{mathptmx}

% Indent
\usepackage{indentfirst}
\setlength{\parindent}{0.3in}

% Title section
\pagestyle{fancy}
\thispagestyle{empty}
\renewcommand{\headrulewidth}{0pt}
\newcommand\papertitle[1]{{\centering\fontsize{20pt}{1em}\textsc{#1}\\\mbox{}\\}
\fancyhead[OC]{\fontsize{12pt}{12pt}\selectfont\textit{#1}}}
\newcounter{people}
\newcommand\paperauthortext[4]{{\centering\fontsize{12pt}{1em}\selectfont
\textsc{#1. #2}\\[-0.1em]{\fontsize{9pt}{1em}\selectfont\textit{\vspace{-1.25em}
\phantom{.}\\#3}}\\\mbox{}\\
\fancyhead[EC]{\fontsize{12pt}{12pt}\selectfont\textit{#4}}}}
\newcommand\paperauthor[3]{{\stepcounter{people}
\ifnum\value{people}=1
{\paperauthortext{#1}{#2}{#3}{#1. #2}
\global\def\auth{#2\xspace}}
\else\ifnum\value{people}=2
{\paperauthortext{#1}{#2}{#3}{\auth and #2}}
\else{\paperauthortext{#1}{#2}{#3}{\auth et al}}\fi\fi}}
\newcommand\paperdate[1]{{\centering\fontsize{9pt}{1em}\selectfont\text{
(Received #1)}\\[2em]}}

% Page header
\newcommand{\paperheader}[1]{\fancyhead[EC]{\fontsize{12pt}{12pt}\selectfont
\textit{#1}}}
\fancyhead[RO, EL]{\fontsize{12pt}{12pt}\selectfont\thepage}
\cfoot{}

\makeatletter
\newenvironment{paperadjustwidth}[2]{
  \begin{list}{}{
    \setlength\partopsep\z@
    \setlength\topsep\z@
    \setlength\listparindent\parindent
    \setlength\parsep\parskip
    \linespread{0}\selectfont
    \@ifmtarg{#1}{\setlength{\leftmargin}{\z@}}
                 {\setlength{\leftmargin}{#1}}
    \@ifmtarg{#2}{\setlength{\rightmargin}{\z@}}
                 {\setlength{\rightmargin}{#2}}
    }
    \item[]}{\end{list}}
\makeatother

% Abstract environment
\newenvironment{paperabstract}
{\begin{paperadjustwidth}{0.5in}{0.5in}\bgroup\fontsize{9pt}{1em}\selectfont
\hspace{0.5in}}
{\egroup\end{paperadjustwidth}}

% Report environment
\setlength\columnsep{0.5in}
\newenvironment{paper}
{\begin{multicols*}{2}\bgroup\fontsize{12pt}{13.8pt}\selectfont}
{\egroup\end{multicols*}}

%Sources
\newsavebox{\sourcebox}
\newcommand{\papersource}[1]{
\vspace{-2em}
\text{}\\*
\fontsize{9pt}{10.35}\selectfont
\noindent\renewcommand{\labelenumi}{}
\savebox{\sourcebox}{\parbox{3in}{\begin{enumerate}
%\vspace{-\baselineskip}
\setlength{\leftmargini}{-1ex}
\setlength{\leftmargin}{-1ex}
\setlength{\labelwidth}{0pt}
\setlength{\labelsep}{0pt}
\setlength{\listparindent}{0pt}
\item\textit{\hspace{-0.35in}#1}
\end{enumerate}}}
\usebox{\sourcebox}
}

% Section headers
\newcounter{papersectioncounter}
\newcounter{papersubsectioncounter}[papersectioncounter]
\newcommand\papersection[1]{
\stepcounter{papersectioncounter}
{\begin{center}\Roman{papersectioncounter} \textsc{#1}\end{center}}}
\newcommand\papersubsection[1]{\stepcounter{papersubsectioncounter}
{\begin{center}\Roman{papersectioncounter}.\Roman{papersubsectioncounter}
\textsc{#1}\\[0.5em]\end{center}}}

%equation
\newcounter{paperequationcounter}
\newcommand\paperequation[3]{{
\stepcounter{paperequationcounter}
\mbox{}\vspace{-0.75em}
\begin{equation*}
#1
\tag*{\fontsize{12pt}{1em}\selectfont
$\begin{array}{r}
\cr{\text{[\arabic{paperequationcounter}]}}
\cr{\fontsize{9pt}{1em}\selectfont\textit{\ifx\\#2\\~\else(\fi#2\ifx\\#2\\~
\else)\fi}}
\end{array}$}
\end{equation*}
}
\expandafter\edef\csname eq#3\endcsname{[\arabic{paperequationcounter}]\noexpand
\xspace}}

% Where
\newcommand{\paperwherevar}[3]{&$#1$ & #2 \ifx\\#3\\~\else($\smash{\text{\si{\fi
#3\ifx\\#3\\~\else}}}$)\fi\\}
\newenvironment{paperwhere}
{\bgroup\fontsize{9pt}{9pt}\selectfont Where:\vspace{2pt}\\\begin{tabular}
{rr@{ = }p{2.42in}}}
{\end{tabular}\egroup\vspace{5pt}}

% Tables
\definecolor{LineGray}{gray}{0.5}
\newtabulinestyle{outer=2.25pt LineGray}
\newtabulinestyle{inner=0.75pt LineGray}
\tabulinesep=1.5pt

\newcommand{\paperiline}[0]{\tabucline[inner]{-}}
\newcommand{\paperoline}[0]{\tabucline[outer]{-}}

% Index column type
\newcolumntype{I}{X[-5,c]}
% Column type with uncertainty
\newcolumntype{U}{@{}X[-5,r]@{$\pm$}X[-5,l]@{}}
% Column type without uncertainty
\newcolumntype{C}{@{}X[-5,c]@{}}

\newcounter{papertableindexcounter}
\newcommand{\papertableindexheader}[0]{\multirow{2}{*}{\textsc{Index}}}
\newcommand{\papertableindex}[0]{\stepcounter{papertableindexcounter}
\arabic{papertableindexcounter}}
\newcommand{\papertableuheadersymbol}[1]{&\multicolumn{2}{c|[inner]}{$#1$}}
\newcommand{\papertableuheadersymbole}[1]{&\multicolumn{2}{c|[outer]}{$#1$}}
\newcommand{\papertableuheaderunit}[1]{&\multicolumn{2}{c|[inner]}{(#1)}}
\newcommand{\papertableuheaderunite}[1]{&\multicolumn{2}{c|[outer]}{(#1)}}

\newcommand{\papertablecheadersymbol}[1]{&$#1$}
\newcommand{\papertablecheaderunit}[2]{&($\pm$#1 #2)}

% Value in table with uncertainty.
\newcommand{\papertableuval}[2]{& #1 & #2}
% Value in table without uncertainty.
\newcommand{\papertablecval}[1]{& #1}

\newenvironment{papertable}[1]
{\setcounter{papertableindexcounter}{0} 
\begin{tabu} to \linewidth {#1}}
{\end{tabu}\vspace{12pt}}

%Figure counter
\newcounter{paperfigurecounter}
\newcommand{\papercaption}[1]{
\vspace{-24pt}
\bgroup
\stepcounter{paperfigurecounter}
\captionof{figure}{\fontsize{9pt}{9pt}\selectfont
\hspace{0.3in}Fig \arabic{paperfigurecounter}. #1}\egroup}

\newcommand{\paperaxis}[9]
{title=#1,
axis x line = bottom,
xmin=#4,xmax=#6,
axis y line = left,
ymin=#5,ymax=#7,
height = 180pt,
grid=both,
x axis line style=-,
y axis line style=-,
x tick label style={
/pgf/number format/.cd,
fixed,
fixed zerofill,
precision=#8,
/tikz/.cd
},
y tick label style={
/pgf/number format/.cd,
fixed,
fixed zerofill,
precision=#9,
/tikz/.cd
}
}
\newcommand{\paperaxisxlabel}[2]{
xlabel=\fontsize{10pt}{1em}\selectfont#1$(#2)\rightarrow$}
\newcommand{\paperaxisylabel}[2]{
ylabel=\fontsize{10pt}{1em}\selectfont#1$(#2)\rightarrow$}
\newcommand{\papergraphoutline}[4]{
\addplot [mark=none,line width=0.75pt] coordinates {
(#1,#2)
(#1,#4)
(#3,#4)
(#3,#2)
(#1,#2)
};}

\newenvironment{papergraph}{
\begin{tikzpicture}
\begin{axis}
}
{\end{axis}
\end{tikzpicture}}

\newcommand{\abs}[1]{\left\lvert#1\right\rvert}
\newcommand{\oo}[0]{\infty}
\newcommand{\sigmaSum}[3]{\sum\limits_{#1}^{#2} #3}
\newcommand{\limto}[3]{\lim\limits_{#1\rightarrow#2}#3}
\renewcommand{\d}[0]{\mathrm{d}}
\newcommand{\cross}[0]{\times}
\newcommand{\lp}{\left(}
\newcommand{\rp}{\right)}
\newcommand\pars[1]{\lp#1\rp}
\newcommand\sqbrack[1]{\left[#1\right]}
\newcommand\R{\mathbb{R}}
\newcommand\di{\partial}
\newcommand\x{\times}
\newcommand\del{\nabla}

\diagrams
\theorems
\begin{document}
\papertitle{A Database of Ribbon Knots in Tangle Form}
%\paperauth{D}{Bar-Natan}{}
\paperauth{A}{Khesin}{}
\paperdate{TODO DATE}
\begin{paperabs}
There are 21 ribbon knots with 10 crossings or fewer.
All 21 of these knots can be expressed as symmetric unions (see \cite{many} and
\cite{one}).
We show that for each of these ribbon knots, we can find a tangle with 4 strands
that satisfies the property that a specific closure of that tangle results in
the corresponding knot while a second closure results in the 2-component unlink.
We show that a tangle satisfying similar conditions exists for any ribbon knot
and that the knot corresponding to such a tangle is always a ribbon knot.
We also provide diagrams and the tangle information of the first 21 ribbon
knots.
\end{paperabs}
\begin{paper}
\papersec{Introduction}

There are 250 knots with 10 crossings or fewer.
Of these, 21 satisfy a particular set of conditions to be deemed \textit{ribbon
knots}.
In the Rolfsen Table, these knots have been given the indices $6_1$, $8_8$,
$8_9$, $8_{20}$, $9_{27}$, $9_{41}$, $9_{46}$, $10_3$, $10_{22}$, $10_{35}$,
$10_{42}$, $10_{48}$, $10_{75}$, $10_{87}$, $10_{99}$, $10_{123}$, $10_{129}$,
$10_{137}$, $10_{140}$, $10_{153}$, and $10_{155}$.
It has been shown that these 21 knots can be expressed as \textit{symmetric
unions} (see \cite{many} and \cite{one}).
A symmetric union is a knot that is symmetric on either side of a central axis,
with the exception of crossings located on this axis.
These symmetric unions can be used to construct presentations of these ribbon
knots that are mathematically interesting.

\paperfig{Singularities}{\svgsize{clasp}{0.4\columnwidth}\hfill
\svgsize{ribbon}{0.4\columnwidth}\\

\noindent Clasp Singularity\hfill Ribbon Singularity}
{A clasp singularity and a ribbon singularity.
A ribbon singularity only has strands that intersect the disc of the knot in
pairs.
In other words, the curve formed by the intersection is entirely contained
within the disc and does not cross its boundary.
A knot that contains exclusively ribbon singularities is called a ribbon knot.}

A ribbon knot is a knot that contains only ribbon singularities (see
\figSingularities).
A ribbon singularity occurs when the self-intersections of the disk bounding the
knot come in pairs.
Thus, when the boundary of the disk passes through the disk, there will be a
second such intersection, but with the opposite sign, where the boundary crosses
the surface of the disk in opposite direction.

\papersvg{Presentation}{presentation}{The ribbon presentation of $6_1$.
Note that there are two rectangular components which are the images of two
discs.
The ribbon connecting them is the unzipped image of a strand connecting the
discs.
Since the only intersections in a ribbon presentations are between the ribbon
and the components, any ribbon presentation clearly forms a ribbon knot.}

A \textit{ribbon presentation} of a ribbon knot depicts the knot as the image of
a series of disks, each connected to the next by an arc, the ribbon, which is
unzipped into two antiparallel strands after applying the mapping that sends the
discs to the knot (see \figPresentation).

\papersec{Tangles}

A \textit{tangle} is a collection of crossings with a certain number of loose
ends.
An $n$-strand tangle will have $2n$ such ends.

\paperfig{Tangle}{\size{9}{
\hspace{0.19\columnwidth}$1_t$\hspace{0.16\columnwidth}$2_t$
\hspace{0.04\columnwidth}\dots\hspace{0.02\columnwidth}$(n-1)_t$
\hspace{0.105\columnwidth}$n_t$\\
\svgc{tangle}\\
\indent\hspace{0.19\columnwidth}$1_b$\hspace{0.16\columnwidth}$2_b$
\hspace{0.04\columnwidth}\dots\hspace{0.02\columnwidth}$(n-1)_b$
\hspace{0.09\columnwidth}$n_b$}}
{An $n$-strand tangle.
In a tangle, the ends do not have to be drawn along the top and bottom of the
tangle; this is done for convenience.
The $2n$ ends could be located anywhere on the perimeter.
We label the ends of an $n$-strand tangle from $1_t$ to $n_t$ along the top and
$1_b$ to $n_b$ along the bottom.
The central R represents a knot that shows that the strands are tangled, but
link up with their corresponding ends the way they do in a pure braid.}

For convenience, a tangle is often drawn like a pure braid, with $n$ ends along
both the top and the bottom where the order of the strands along the top is the
same as along the bottom.
Thus, if we were to follow each strand in a tangle, we would end up below our
starting point.
We denote the tops of the strands by $1_t,~2_t,~\dots,~n_t$, and the
bottoms  by $1_b,~2_b,~\dots,~n_b$ (see \figTangle).

A \textit{closure} of a tangle is a way of stitching the ends of the tangle
together to reduce the resulting number of strands.

\paperfig{Top}{\size{9}{
\hspace{0.1\columnwidth}$1_t$\hspace{0.07\columnwidth}$2_t$
\hspace{0.07\columnwidth}$3_t$\hspace{0.128\columnwidth}\dots
\hspace{0.04\columnwidth}$(2n-2)_t$ $(2n-1)_t$ $(2n)_t$\\
\svgc{top}\\
\indent\hspace{0.1\columnwidth}$1_b$\hspace{0.07\columnwidth}$2_b$
\hspace{0.06\columnwidth}$3_b$\hspace{0.12\columnwidth}\dots
\hspace{0.04\columnwidth}$(2n-2)_b$ $(2n-1)_b$ $(2n)_b$}}
{A top closure of a $2n$-strand tangle.
The top closure leaves every bottom end open and leaves all the upper ends
closed.
After the closure, the tangle has only $n$ strands instead of $2n$.
Only 4 of the $n$ closures are shown here.}

\begin{paperdef}{TopClosure}{For a $2n$-strand tangle, we define the
\textit{top closure} of a tangle to be the closure that stitches together the
pairs $(1_t,~2_t),~(3_t,~4_t),~\dots,~((2n-1)_t,~(2n)_t)$ to create
$n$ strands (see \figTop).}\end{paperdef}

\paperfig{Full}{\size{9}{
\hspace{0.1\columnwidth}$1_t$\hspace{0.07\columnwidth}$2_t$
\hspace{0.07\columnwidth}$3_t$\hspace{0.128\columnwidth}\dots
\hspace{0.04\columnwidth}$(2n-2)_t$ $(2n-1)_t$ $(2n)_t$\\
\svgc{full}\\
\indent\hspace{0.1\columnwidth}$1_b$\hspace{0.07\columnwidth}$2_b$
\hspace{0.06\columnwidth}$3_b$\hspace{0.12\columnwidth}\dots
\hspace{0.04\columnwidth}$(2n-2)_b$ $(2n-1)_b$ $(2n)_b$}}
{A full closure of a $2n$-strand tangle.
It is important to note that the full closure reproduces the top closure along
the bottom ends, and closes the top ends similarly, but offset by 1 strand.
This closure leaves ends $1_t$ and $(2n)_t$ open.
After the closure, the tangle has only 1 strand instead of $2n$.
Only 6 of the $2n-1$ closures are shown here fully.
The break in the central top closure represents the presence of more closed
strands.}

\begin{paperdef}{FullClosure}{For a $2n$-strand tangle, we define the
\textit{full closure} of a tangle to be the closure that stitches together the
pairs $(1_b,~2_b),~(2_t,~3_t),~(3_b,~4_b)$, \dots,
$((2n-2)_t,~(2n-1)_t),~((2n-1)_b,~(2n)_b)$ to create 1 strand (see
\figFull).}\end{paperdef}

If the inside of the tangle is trivial, then in the top closure, the strands
create an untangle which we can turned into the unlink by closing the strands
along the bottom as depicted in the full closure (see \figFull).

These closures allow us to show an important property of ribbon knots.

\begin{paperthm}{Ribbon}
$K$ is a ribbon knot$\iff\exists$ a $2n$-strand tangle $T$ s.~t.~the top closure
of $T$ creates the $n$-strand untangle and the full closure of $T$ creates a
tangle with 1 strand, identical to the knot $K$ when $(1_t,~(2n)_t)$ is closed.
\end{paperthm}
\begin{proof}
We know that the top closure of $T$ effectively results in $n$ ribbons that can
be completely untangled (see \figTop).
Since we can untangle the ribbons, the only self-intersections that the surface
of the tangle can have are going to be ribbon singularities, as we can only get
ribbon singularities by undoing the untangling process.\\

\papersvg{Proof}{proof}
{The result of top closing a tangle, closing it symmetrically
along the bottom, and then connecting each resulting unlink to the next using a
ribbon.
Connecting the unlinks involves making the connections
$(2_t,~3_t),~(4_t,~5_t),~\dots,~((2n-2)_t,~(2n-1)_t)$ with a pair of strands.
Note that this results in one long strand running along the top of the knot, so
these connections are just the full closure of the tangle along with the
connection $(1_t,~(2n)_t)$.}

Now we can close those $n$ ribbons along the bottom to create $n$ unknots, which
are the $n$-component unlink.
Then we can connect each unknot to the next with a very thin ribbon by
connecting the pairs $(2_t,~3_t),~(4_t,~5_t),~\dots,~((2n-2)_t,~(2n-1)_t)$ with
a pair of strands each.
These ribbons might pass through the surface of the knot, but since they are
ribbons, this would only add ribbon singularities to the knot.
Since we had top closed the knot earlier, what we have effectively done is
connected each top strand to the next, starting from $1_t$ and ending with
$(2n)_t$ (see \figProof).
The remaining connections make the full closure (see \figFull).
However, connecting each top strand to the next by connecting
$(1_t,~2_t),~(3_t,~4_t),~\dots,~((2n-1)_t,~(2n)_t)$ is the same as connecting
$(1_t,~(2n)_t)$.
This means that we have added the one connection to the full closure to create
the knot $K$ and we have shown that the resulting knot only contains ribbon
singularities.
Therefore, $K$ must be a ribbon knot.

Conversely, consider an $n$-strand tangle $T$.
We can place all $2n$ of its ends along the top numbered $1_t$ to $(2n)_t$.\\

\paperfig{Lemma}{\svgsize{lemmaone}{0.4\columnwidth}\\

\vspace{-4em}\hspace{0.3in}\hspace{15.2ex}\size{20}{=}\\

\vspace{-4.3em}\hfill\svgsize{lemmatwo}{0.4\columnwidth}\\

\hspace{0.3in}\hspace{15.2ex}\size{20}{$\Downarrow$}\\

\svgsize{lemmathree}{0.4\columnwidth}\\

\vspace{-4em}\hspace{0.3in}\hspace{15.2ex}\size{20}{=}\\

\vspace{-4.3em}\hfill\svgsize{lemmafour}{0.4\columnwidth}\\}
{A visual representation of an implication involving tangles.
The top equality shows that the tangle T can be top closed to
form the unlink, the result of top closing the untangle U.
The implication states that there must exist a tangle T' that can be closed
along the bottom as shown to create T.}

\begin{paperlem}{Tangles}
If an $n$-strand tangle $T$ with all $2n$ of its ends placed along the top can
be top closed to create the $n$-component unlink, then $\exists$ a $2n$-strand
tangle $T'$ s.~t.~applying the bottom part of the full closure results in $T$
(see \figLemma).
\end{paperlem}

\begin{proof}[Proof of Lemma]
We know that after top closing $T$, we can apply a continuous spatial
deformation to turn it into the unlink.
Once we have untangled $T$, we can attach a strand to each of the resulting
unknots.
We can extend this strand to below the boundary of the tangle and fix it to some
arbitrary point.
If we were to reverse the earlier spatial deformation, the parts of the strands
that we added that were outside of the tangle would remain there, while the
remainder would be tangled up inside $T$.
We can now unzip these strands into ribbons, and think of them as part of the
tangle, resulting in a new path for each strand of $T$.
Thus, these new paths follow their old paths, go on a long detour until they
reach a point outside of the tangle, turn around and come back, and then resume
their old path.

We see that by aligning the loops that these paths trace along the bottom,
we can view them as closures to a series of strands.
Thus, with these paths the tangle is $T$, but without these paths it is $T'$,
as when this $T'$ is closed along the bottom by these loops, we get long ribbons
that extend from the strands of $T$, but are nevertheless equivalent to $T$.
Therefore, such a $T'$ exists.
\end{proof}

\papersvg{Lowered}{lowered}
{A representation of a knot in ribbon presentation after all of the
intersections were enclosed in one large tangle labeled R.
The components of the knots are depicted exactly as they are while the
intersections of the ribbons and the components are not shown.}

For any knot in its ribbon presentation, we can enclose all of the lower halves
of the components in one large tangle (see \figPresentation).
Next, we can move all of the ribbons so that all of the connections between
ribbons and components occur outside of this tangle but any intersections occur
inside (see \figLowered).\\

\papersvg{Twisted}{twisted}
{A representation of a knot in ribbon presentation after all of the
intersections were enclosed in one large tangle labeled R and all of the ribbons
were extracted out of it.
The ribbons of the knots are depicted exactly as they are while the
intersections of the components are not shown.}

After we have enclosed the knot in a tangle, we can pull all the ribbons out of
this tangle.
However, the tangle was probably non-trivial since the ribbons could intersect
the components.
Luckily, all of these intersections form ribbon singularities, so we can move
them back into the tangle.

If a ribbon passed through a component before we pulled it up, there will be a
strand on either side of the ribbon that we can move back into the tangle as the
area below and above the ribbons is freely connected to the tangle (see
\figTwisted).

However, since the top part of the knot no longer has any intersections, we can
see that it is equivalent to the top half of the full closure of the large
tangle, as well as the connection $(1_t,~(2n)_t)$ (see \figFull).

We know that since every ribbon knot has a ribbon presentation, then removing
the ribbons that connect the components of the knot results in the $n$-component
unlink (see \figPresentation).
We also know that if were to take our constructed tangle that encloses all of
the crossings of our knot and top close the tangle, the result would be very
similar the full closure that we constructed earlier (see \figTop).
The reason for this is that the top closure would close each component with its
other end, effectively removing all of the connecting ribbons (see \figTwisted).

Thus, the top closure of our tangle forms the unlink, as it removes all the
ribbons from a ribbon knot.
This means that this tangle satisfies the conditions of $T$ in \lemTangles.
As a consequence, we know that there must exist a tangle $T'$ that can be closed
along the bottom to result in $T$.

We know that this $T'$ can be top closed to result in the untangle, as $T$ can
be top closed to result in the unlink (see \figLemma).
Additionally, if the bottom half of the full closure of $T'$ results in $T$ and
the top half of the full closure with the connection $(1_t,~(2n)_t)$ of $T$
results in our knot, then the full closure of $T'$ with the same connection also
results in our knot.

Therefore, this tangle given to us by \lemTangles satisfies the conditions of
the theorem and we have shown its existence.
\end{proof}

\papersec{Symmetric Unions}

It is known that all 21 ribbon knots with 10 crossings or fewer can be drawn as
symmetric unions (see \cite{many} and \cite{one}).
This allowed us to find a tangle satisfying the conditions of \thmRibbon for
each of these 21 ribbon knots.
This was possible due to the fact that adjusting a few connections in a
symmetric union makes it untwist and untangle as we would like it to after
applying the top closure (see \figTop).
Surprisingly, each of these tangles only had 2 components, despite the fact that
there is no requirement stipulating that this must be the case.

\begin{paperqtn}{Components}
Do all the tangles for ribbon knots that are generated by \thmRibbon have 2
components?
\end{paperqtn}

We present these tangles for all 21 of the ribbon knots, drawn with the full
closure and the connection $(1_t,~4_t)$ added, effectively recreating the
original knots.
\end{paper}

\setlength{\tabcolsep}{12pt}
\begin{tabular}{cccc}
\svgsize{6_1}{0.17\columnwidth}&\svgsize{8_8}{0.17\columnwidth}&
\svgsize{8_9}{0.17\columnwidth}&\svgsize{8_20}{0.17\columnwidth}\\
$6_1$&$8_8$&$8_9$&$8_{20}$\\
&&&\\
\svgsize{9_27}{0.17\columnwidth}&\svgsize{9_41}{0.17\columnwidth}&
\svgsize{9_46}{0.17\columnwidth}&\svgsize{10_3}{0.17\columnwidth}\\
$9_{27}$&$9_{41}$&$9_{46}$&$10_3$\\
&&&\\
\svgsize{10_22}{0.17\columnwidth}&\svgsize{10_35}{0.17\columnwidth}&
\svgsize{10_42}{0.17\columnwidth}&\svgsize{10_48}{0.17\columnwidth}\\
$10_{22}$&$10_{35}$&$10_{42}$&$10_{48}$\\
&&&\\
\svgsize{10_75}{0.17\columnwidth}&\svgsize{10_87}{0.17\columnwidth}&
\svgsize{10_99}{0.17\columnwidth}&\svgsize{10_123}{0.17\columnwidth}\\
$10_{75}$&$10_{87}$&$10_{99}$&$10_{123}$\\
&&&\\
\svgsize{10_129}{0.17\columnwidth}&\svgsize{10_137}{0.17\columnwidth}&
\svgsize{10_140}{0.17\columnwidth}&\svgsize{10_153}{0.17\columnwidth}\\
$10_{129}$&$10_{137}$&$10_{140}$&$10_{153}$\\
&&&\\
\svgsize{10_155}{0.17\columnwidth}&&&\\
$10_{155}$&&&
\end{tabular}

\begin{paper}
The locations of the cuts that result in the ends $1_b$ and $2_b$ as well as
$3_b$ and $4_b$ are, in general, not very interesting.
It is merely a location on the knot that can be kept on the outside while
untangling the top closure of the tangle.
However, in the symmetric union, the locations of the two other cuts can always
be obtained by symmetrically deforming the knot until one of the bridges appears
above any crossings on the axis of symmetry.
Then the knot can be cut just above this bridge along the strands that cross it
closest to the axis.

\newsavebox{\mirrorR}
\sbox{\mirrorR}{\reflectbox{\size{9}{R}}}

\svgc{question}

\papercap{Question}{Here, R represents a tangle whose mirror image is
\usebox{\mirrorR}.
They are connected by X, a series of crossings and two bridges.
U is the untangle, closed to create the unlink.
In the lower diagram, the knot has been cut and restitched just outside of the
upper bridge.
The lower strand was pulled under the bridge after being stitched.}

\begin{paperqtn}{Cuts}
For any ribbon knot in its symmetric union presentation, can the location of the
cuts between $(1_t,~4_t)$ and $(2_t,~3_t)$ be generalized to be along the
two strands that are first to cross one of the 2 bridges of the symmetric union?
\end{paperqtn}

\papersec{Database}

A database of the planar diagram crossing information of the 21 ribbon
knots has been created.
This notation consists of a list of crossings of the form $X_{a,b,c,d}$ which
labels the four strands in a crossing from the lower incoming strand and
continues counterclockwise.

\begin{center}\svgsize{crossing}{0.5\columnwidth}\end{center}

\papercap{Crossing}{This shows a typical crossing.
The numbering stars from the lower incoming segment and proceeds
counterclockwise.
Thus, this crossing would be labeled $X_{a,b,c,d}$.}

The knots are encoded as they appear in the table above.
The tangle information consists of the list of segments of the knot of each
strand of the tangle before any stitching between crossings is done.
A strand in a crossing will carry the number of the knot segment going into it.
Here is knot $6_1$ as an example.

\svgc{example}

\papercap{Example}{This depicts all the segments of knot $6_1$ as numbered.
The numbering always starts from strand 1 and proceeds along the knot.
The planar diagram crossing information of $6_1$ is $X_{2,8,3,7}$
$X_{5,14,6,15}$ $X_{8,20,9,19}$ $X_{3,10,4,11}$ $X_{11,6,12,7}$ $X_{15,4,16,5}$
$X_{17,16,18,17}$ $X_{18,10,19,9}$ $X_{20,2,21,1}$ $X_{21,12,22,13}$
$X_{22,14,1,13}$.
Meanwhile, the tangle information of $6_1$ is \{\{1, 2, 3, 4, 5, 6, 7, 8\},
\{12, 11, 10, 9\}, \{13, 14, 15, 16\}, \{22, 21, 20, 19, 18, 17\}\}.}

The full database may be accessed at TODO URL.

\papersec{References}

\begin{thebibliography}{}
\bibitem{one}
Lamm, Christoph.
\textit{Symmetric union presentations for 2-bridge ribbon knots.}
\texttt{arXiv:math/0602395}
\bibitem{many}
Lamm, Christoph.
\textit{Symmetric unions and ribbon knots.}
Osaka J. Math. 37 (2000), no. 3, 537--550
\end{thebibliography}

\papersec{Acknowledgements}

\end{paper}
\end{document}
