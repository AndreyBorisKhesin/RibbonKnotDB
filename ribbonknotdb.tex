\documentclass[twoside]{article}
\usepackage{amsmath}
\usepackage{amssymb}
\usepackage{amsthm}
\usepackage{capt-of}
\usepackage{caption}
\captionsetup{labelformat=empty,labelsep=none}
\usepackage[strict]{changepage}
\usepackage{chngcntr}
\usepackage[americanvoltage, siunitx]{circuitikz}
\usepackage{color,colortbl}
\usepackage{fancyhdr}
\usepackage[T1]{fontenc}
\usepackage{gensymb}
\usepackage[margin=1in]{geometry}
\usepackage{graphicx}
\usepackage{import}
\usepackage{indentfirst}
\usepackage{mathptmx}
\usepackage{mathrsfs}
\usepackage{multicol}
\usepackage{multirow}
\usepackage{pgfplots}
\usepackage{pgfplotstable}
\usepgfplotslibrary{external}
\usepackage{siunitx}
\usepackage{tabu}
\usepackage{tikz}
\usetikzlibrary{positioning,matrix,shapes,chains,arrows}
\tikzexternalize[prefix=precompiled_figures/]
\usepackage{xspace}

% Indent
\setlength{\parindent}{0.3in}

\newcounter{papertheoremamount}
\newcommand\theorems[0]{\newtheorem{conjecture}[subsection]{Conjecture}
\newtheorem{corollary}[subsection]{Corollary}
\newtheorem{lemma}[subsection]{Lemma}\newtheorem{remark}[subsection]{Remark}
\newtheorem{theorem}[subsection]{Theorem}
\newtheorem{question}[subsection]{Question}}
\newcommand\subtheorems[0]{\stepcounter{papertheoremamount}
\newtheorem{conjecture}[subsubsection]{Conjecture}
\newtheorem{corollary}[subsubsection]{Corollary}
\newtheorem{lemma}[subsubsection]{Lemma}
\newtheorem{remark}[subsubsection]{Remark}
\newtheorem{theorem}[subsubsection]{Theorem}
\newtheorem{question}[subsubsection]{Question}}

% Title section
\pagestyle{fancy}
\thispagestyle{empty}
\renewcommand{\headrulewidth}{0pt}
\newcommand\papertitle[1]{{\centering\fontsize{20pt}{1em}\textsc{#1}\\\mbox{}\\}
\fancyhead[OC]{\fontsize{12pt}{12pt}\selectfont\textit{#1}}}
\newcounter{people}
\newcommand\paperauthortext[4]{{\centering\fontsize{12pt}{1em}\selectfont
\textsc{#1. #2}\\[-0.1em]{\fontsize{9pt}{1em}\selectfont\textit{\ifx&#3&
\vspace{-1em}\else#3\fi}}\\\mbox{}\\
\fancyhead[EC]{\fontsize{12pt}{12pt}\selectfont\textit{#4}}}}
\newcommand\paperauthor[3]{{\stepcounter{people}
\ifnum\value{people}=1
{\paperauthortext{#1}{#2}{#3}{#1. #2}
\global\def\auth{#2\xspace}}
\else\ifnum\value{people}=2
{\paperauthortext{#1}{#2}{#3}{\auth and #2}}
\else{\paperauthortext{#1}{#2}{#3}{\auth et al}}\fi\fi}}
\newcommand\paperdate[1]{{\centering\fontsize{9pt}{1em}\selectfont\text{
(Received #1)}\\[2em]}}

% Page header
\newcommand{\paperheader}[1]{\fancyhead[EC]{\fontsize{12pt}{12pt}\selectfont
\textit{#1}}}
\fancyhead[RO, EL]{\fontsize{12pt}{12pt}\selectfont\thepage}
\cfoot{}

\makeatletter
\newenvironment{paperadjustwidth}[2]{
  \begin{list}{}{
    \setlength\partopsep\z@
    \setlength\topsep\z@
    \setlength\listparindent\parindent
    \setlength\parsep\parskip
    \linespread{0}\selectfont
    \@ifmtarg{#1}{\setlength{\leftmargin}{\z@}}
                 {\setlength{\leftmargin}{#1}}
    \@ifmtarg{#2}{\setlength{\rightmargin}{\z@}}
                 {\setlength{\rightmargin}{#2}}
    }
    \item[]}{\end{list}}
\makeatother

% Abstract environment
\newenvironment{paperabstract}
{\begin{paperadjustwidth}{0.5in}{0.5in}\bgroup\fontsize{9pt}{1em}\selectfont
\hspace{0.5in}}
{\egroup\end{paperadjustwidth}}

% Report environment
\setlength\columnsep{0.5in}
\newenvironment{paper}
{\begin{multicols*}{2}\bgroup\fontsize{12pt}{13.8pt}\selectfont}
{\egroup\end{multicols*}}

%Sources
\newsavebox{\sourcebox}
\newcommand{\papersource}[1]{
\vspace{-2em}
\text{}\\*
\fontsize{9pt}{10.35}\selectfont
\noindent\renewcommand{\labelenumi}{}
\savebox{\sourcebox}{\parbox{3in}{\begin{enumerate}
%\vspace{-\baselineskip}
\setlength{\leftmargini}{-1ex}
\setlength{\leftmargin}{-1ex}
\setlength{\labelwidth}{0pt}
\setlength{\labelsep}{0pt}
\setlength{\listparindent}{0pt}
\item\textit{\hspace{-0.35in}#1}
\end{enumerate}}}
\usebox{\sourcebox}
}

%Section headers
\newcounter{papersectioncounter}
\newcounter{papersubsectioncounter}[papersectioncounter]
\newcommand\papersection[1]{\stepcounter{papersectioncounter}
\stepcounter{section}
\begin{center}\Roman{papersectioncounter} \textsc{#1}\end{center}}
\newcommand\papersubsection[1]{\stepcounter{papersubsectioncounter}
\addtocounter{subsection}{\thepapertheoremamount}
\setcounter{subsubsection}{0}
{\begin{center}
\Roman{section}.\Roman{papersubsectioncounter}
\textsc{#1}\\[0.5em]\end{center}}}

%equation
\newcounter{paperequationcounter}
\newcommand\paperequation[3]{{
\stepcounter{paperequationcounter}
\mbox{}\vspace{-0.75em}
\begin{equation*}
#1
\tag*{\fontsize{12pt}{1em}\selectfont
$\begin{array}{r}
\cr{\text{[\arabic{paperequationcounter}]}}
\cr{\fontsize{9pt}{1em}\selectfont\textit{\ifx\\#2\\~\else(\fi#2\ifx\\#2\\~
\else)\fi}}
\end{array}$}
\end{equation*}
}
\expandafter\edef\csname eq#3\endcsname{[\arabic{paperequationcounter}]\noexpand
\xspace}}

% Where
\newcommand{\paperwherevar}[3]{&$#1$ & #2 \ifx\\#3\\~\else($\smash{\text{\si{\fi
#3\ifx\\#3\\~\else}}}$)\fi\\}
\newenvironment{paperwhere}
{\bgroup\fontsize{9pt}{9pt}\selectfont Where:\vspace{2pt}\\\begin{tabular}
{rr@{ = }p{2.42in}}}
{\end{tabular}\egroup\vspace{5pt}}

% Tables
\definecolor{LineGray}{gray}{0.5}
\newtabulinestyle{outer=2.25pt LineGray}
\newtabulinestyle{inner=0.75pt LineGray}
\tabulinesep=1.5pt

\newcommand{\paperiline}[0]{\tabucline[inner]{-}}
\newcommand{\paperoline}[0]{\tabucline[outer]{-}}

% Index column type
\newcolumntype{I}{X[-5,c]}
% Column type with uncertainty
\newcolumntype{U}{@{}X[-5,r]@{$\pm$}X[-5,l]@{}}
% Column type without uncertainty
\newcolumntype{C}{@{}X[-5,c]@{}}

\newcounter{papertableindexcounter}
\newcommand{\papertableindexheader}[0]{\multirow{2}{*}{\textsc{Index}}}
\newcommand{\papertableindex}[0]{\stepcounter{papertableindexcounter}
\arabic{papertableindexcounter}}
\newcommand{\papertableuheadersymbol}[1]{&\multicolumn{2}{c|[inner]}{$#1$}}
\newcommand{\papertableuheadersymbole}[1]{&\multicolumn{2}{c|[outer]}{$#1$}}
\newcommand{\papertableuheaderunit}[1]{&\multicolumn{2}{c|[inner]}{(#1)}}
\newcommand{\papertableuheaderunite}[1]{&\multicolumn{2}{c|[outer]}{(#1)}}
\newcommand{\papertablecheadersymbol}[1]{&$#1$}
\newcommand{\papertablecheaderunit}[2]{&($\pm$#1 #2)}

% Value in table with uncertainty.
\newcommand{\papertableuval}[2]{& #1 & #2}
% Value in table without uncertainty.
\newcommand{\papertablecval}[1]{& #1}

\newenvironment{papertable}[1]
{\setcounter{papertableindexcounter}{0} 
\begin{tabu} to \linewidth {#1}}
{\end{tabu}\vspace{12pt}}

%Figure counter
\newcounter{paperfigurecounter}
\newcommand{\papercaption}[1]{
\vspace{-24pt}
\bgroup
\stepcounter{paperfigurecounter}
\captionof{figure}{\fontsize{9pt}{9pt}\selectfont
\hspace{0.3in}Fig \arabic{paperfigurecounter}. #1}\egroup}

\newcommand{\paperaxis}[9]
{title=#1,
axis x line = bottom,
xmin=#4,xmax=#6,
axis y line = left,
ymin=#5,ymax=#7,
height = 180pt,
grid=both,
x axis line style=-,
y axis line style=-,
x tick label style={
/pgf/number format/.cd,
fixed,
fixed zerofill,
precision=#8,
/tikz/.cd
},
y tick label style={
/pgf/number format/.cd,
fixed,
fixed zerofill,
precision=#9,
/tikz/.cd
}
}
\newcommand{\paperaxisxlabel}[2]{
xlabel=\fontsize{10pt}{1em}\selectfont#1$(#2)\rightarrow$}
\newcommand{\paperaxisylabel}[2]{
ylabel=\fontsize{10pt}{1em}\selectfont#1$(#2)\rightarrow$}
\newcommand{\papergraphoutline}[4]{
\addplot [mark=none,line width=0.75pt] coordinates {
(#1,#2)
(#1,#4)
(#3,#4)
(#3,#2)
(#1,#2)
};}

\newenvironment{papergraph}{
\begin{tikzpicture}
\begin{axis}
}
{\end{axis}
\end{tikzpicture}}

\newcommand{\abs}[1]{\left\lvert#1\right\rvert}
\newcommand{\oo}[0]{\infty}
\newcommand{\sigmaSum}[3]{\sum\limits_{#1}^{#2} #3}
\newcommand{\limto}[3]{\lim\limits_{#1\rightarrow#2}#3}
\renewcommand{\d}[0]{\mathrm{d}}
\newcommand{\cross}[0]{\times}
\newcommand{\lp}{\left(}
\newcommand{\rp}{\right)}
\newcommand\pars[1]{\lp#1\rp}
\newcommand\sqbrack[1]{\left[#1\right]}
\newcommand\R{\mathbb{R}}
\newcommand\di{\partial}
\newcommand\x{\times}
\newcommand\del{\nabla}

\theorems
\begin{document}
\papertitle{A Database of Ribbon Knots in Tangle Form}
\paperauthor{D}{Bar-Natan}{}
\paperauthor{A}{Khesin}{}
\paperdate{TODO DATE}
\begin{paperabstract}
All 21 ribbon knots with 10 crossings or fewer can be expressed as symmetric
unions.
These symmetric unions can be cut in four places to create tangles that can be
closed to produce the unlink.
This can be generalized to the point that any ribbon knot can be turned into
such a tangle.
Conversely, every such tangle can be closed to a ribbon knot.
The tangles for the first 21 knots as well as their information in planar
diagram notation has been calculated.
\end{paperabstract}
\begin{paper}
\papersection{Introduction}

A ribbon knot is a knot that contains only ribbon singularities.
A ribbon singularity occurs when the self-intersections of the disk come in
pairs.
Thus when the boundary of the disk passes through said disk, there will always
be a corresponding intersection with the opposite sign, meaning that the
boundary crosses the plane of the disk in opposite direction.
A ribbon presentation for a knot depicts the knot as the image of a series
of disks, each connected to the next by an arc, the ribbon, which is unzipped
into two strands after applying the map.

A symmetric union is a knot that is symmetric about a central axis, save the
crossings that lie on the axis itself.
It is well known that any symmetric union is a ribbon knot.
However, at this time of this writing it is not yet known whether every ribbon
knot can be expressed as a symmetric union.
It has been shown that all 21 knots with 10 crossings or fewer can be expressed
as symmetric unions (\cite{oneknot} and \cite{manyknots}).
\papersection{Tangles}

A tangle is a collection of crossings with a certain number of loose ends.
A tangle with $n$ strands will have $2n$ such ends.
For convenience, a tangle is often drawn similarly to a pure braid, with $n$
ends along the top and $n$ ends along the bottom where the order of the strands
along the top is the same as along the bottom.
In other words, if one were to follow each strand in a tangle, they would end up
exactly below the point at which they started.
Let the tops of the strands be denoted by $1_t$, $2_t$, \dots, $n_t$, and the
bottoms  by $1_b$, $2_b$, \dots, $n_b$.

\svg{diagrams/tangle}
\papercaption{Tangle}{In a tangle, the ends do not have to be drawn along the
top and bottom of the tangle.
This is merely done for convenience.
In practice, the $2n$ ends could be located anywhere on the perimeter.
Here the central $R$ represents a knot that shows that the strands are tangled,
but link up with their corresponding ends the way they do in a pure braid.}

A closure of a tangle is a manner of stitching the ends of the tangle together
to reduce the resulting number of strands.
For a tangle with $2n$ strands, let the top closure of the tangle be defined as
the closure that stitches together the pairs $(1_t$, $2_t)$, $(3_t$, $4_t)$,
\dots, $((2n-1)_t$, $(2n)_t)$ to create $n$ strands.

\svg{diagrams/top}
\papercaption{Top}{This shows how a tangle with $2n$ strands can be top closed.
This leaves every bottom end open.
Now, this tangle has only $n$ strands instead of $2n$.
Note that only 4 of the $n$ closures are shown here.
Here the central $R$ represents a knot that shows that the strands are tangled,
but link up with their corresponding ends the way they do in a pure braid.}

For a tangle with $2n$ strands, let the full closure of the tangle be defined as
the closure that stitches together the pairs $(1_b$, $2_b)$, $(2_t$, $3_t)$,
$(3_b$, $4_b)$, \dots, $((2n-2)_t$, $(2n-1)_t)$, $((2n-1)_b$, $(2n)_b)$ to
create 1 strand.

\svg{diagrams/full}
\papercaption{Full}{This shows how a tangle with $2n$ strands can be full
closed.
It is important to note that this reproduces the top closure along the bottom,
and closes the top similarly, but offset by 1 strand.
This leaves ends $1_t$ and $(2n)_t$ open.
Now, this tangle has only 1 strand instead of $2n$.
Notice that only 8 of the $2n-1$ closures are shown here.
The break in the central top closure represents the presence of more closed
strands.
Here the central $R$ represents a knot that shows that the strands are tangled,
but link up with their corresponding ends the way they do in a pure braid.}

Note that in both scenarios, one is effectively closing the strands along the
bottom in the same manner.
If $R$ is trivial, then in the top closure, the strands create an untangle which
can be turned into the unlink by closing the strands along the bottom as
depicted in the full closure.

\begin{papertheorem}{Ribbon}
A knot $K$ is ribbon$\iff\exists$ a tangle $T$ with $2n$ strands where the
top closure of $T$ creates the trivial tangle with $n$ strands, while
full closure of $T$ creates a tangle with 1 strand, identical to knot $K$ when
$(1_t$, $(2n)_t)$ is joined.
\end{papertheorem}
\begin{proof}
If the top closure of a tangle $T$ results in $n$ trivial strands that can be
untangled, that means that among the crossings in the top closure of $T$, there
are only ribbon singularities, as any clasp singularities would not make the top
closure of $T$ trivial.
Now those $n$ strands can be closed along the bottom to create $n$ unknots.
Then each unknot is connected to the next using a ribbon.
This ribbon may pass through the plane of the original knot, but since a ribbon
consists of 2 strands, this would result in only ribbon singularities.
This has effectively connected every adjacent pair of links as well as the first
and last links together, creating the original knot by the assumption.
Since the resultant knot is $K$ and contains only ribbon singularities, $K$ is
ribbon.

Conversely, consider a tangle $T$ with $n$ strands and $2n$ ends along the top
numbered $1_t$ to $(2n)_t$.
\begin{paperlemma}{Tangles}
If a tangle $T$ with $n$ strands with $2n$ ends along the top can be top closed
with $n$ stitchings to create the unlink consisting of $n$ unknots, then
$\exists$ a tangle $T'$ consisting of $2n$ strands with $2n$ ends along the top
and $2n$ ends along the bottom such that applying the just the bottom part of
the full closure, consisting of the stitchings $(1_b$, $2_b)$, $(3_b$, $4_b)$,
\dots, $((2n-1)_b$, $(2n)_b)$, results in $T$.

\svg{diagrams/lemma}
\papercaption{Lemma}{The top equality shows that the tangle $T$ can be top
closed to form the unlink $U$.
The implication states that there then must exist a tangle $T'$ that can be
closed along the bottom as shown to create $T$.
The break in the central top and bottom closures represents the presence of more
closed strands.}
\end{paperlemma}
\begin{proof}[Proof of Lemma]
Once $T$ is closed, a continuous deformation of three dimensional space turns it
into the unlink by assumption.
Once untangled, a ribbon can be attached to each link, extending below the
tangle.
This can be done trivially as $T$ is untangled.
If the earlier spatial deformation were to be reversed, the ribbons would be
deformed along with $T$.
Now, it suffices to cap the ribbons at the bottom, similar to the bottom part of
a full closure.
The result is a tangle with $2n$ ends along both the top and the bottom where
the bottom ends are closed pairwise as described earlier and the tangle can be
top closed to produce the untangle.
This tangle is therefore $T'$.
\end{proof}
For any knot in its ribbon presentation, the lower half of each component can be
enclosed in one large tangle.
The knot can be deformed until the ribbon connects the tops of the knot's
components, making sure the ends of the ribbons are above the upper boundary of
the tangle.

\svg{diagrams/lowered}
\papercaption{Lowered}{The intersections of the ribbon and the components of the
ribbon knot can be lowered into the enclosed section labeled $R$.
The components of the knots are depicted exactly as they are while the
intersections of the ribbons and the components are not shown.
The break in the central top closure represents the presence of more closed
strands.}

It is important to note that everything above the upper boundary of the tangle
in \figLowered is exactly as described while the ribbons inside the tangle can
have any number of crossings with the components and each other.
Since any space that is in the tangle but above the ribbons is connected to the
outside of the knot, any strands that pass through that space can be moved up
and around the knot to end up fully inside the tangle below the ribbons leaving
nothing above the ribbons.
This means that any ribbon knot with $n$ components can be transformed into a
tangle with $2n$ ends along the top that has been top closed with $n-1$ ribbon
bridges between each of the closing loops.

\svg{diagrams/twisted}
\papercaption{Twisted}{This diagram is very similar to \figLowered but the knot
it represents is very different.
Here, every ribbon, originally twisted with the components of the ribbon knot,
was dragged to the position it occupies here, dragging the strands of the
components with it.
Since every area outside of the enclosed section labeled $R$, whether it be
above the knot or below the ribbons, is connected to $R$, then every strand that
got dragged with the ribbons can be moved to within $R$ once again.
This results in the components inside of $R$ being knotted, whereas before, they
were not.
The break in the central top closure represents the presence of more closed
strands.}

Here the knot in the tangle is completely different and all the crossings are
between components of the knot, as opposed to between ribbons.
This is equivalent to applying just the top part of the full closure to the
tangle and then applying the stitching $(1_t$, $(2n)_t)$.

\svg{diagrams/complete}
\papercaption{Complete}{In this diagram, nothing within $R$ changed from
\figTwisted.
Everything above $R$ was smoothly deformed to illustrate that the resulting
strands above $R$ are simply a top closure and the stitching $(1_t$,
$(2n)_t)$.}

Note that since any ribbon knot is a series of unzipped strands connecting
subsequent components of the unlink, then the removal of these strands will
result in the unlink.
Similarly, if a ribbon knot were to be deformed in the above manner, then
removing the ribbons would still result in the unlink.
Note that in any situation, removing a ribbon is equivalent to unzipping
it, cutting both strands of the ribbon, then stitching each strand to the one it
was unzipped from.

\svg{diagrams/cut}
\papercaption{Cut}{If a ribbon connecting two strands were to be entirely
removed, this would be equivalent to severing both strands of the ribbon and
reattaching each strand to the neighbouring that it was not attached to before.}

Removing the ribbons from the deformed ribbon knot in \figTwisted by repeatedly
applying the process depicted in \figCut would therefore be identical to
applying the top closure to its tangle.
Since applying the top closure to the tangle is the same as removing the
ribbons, which in turn results in the unlink since the knot is ribbon, the
tangle that makes up the bulk of the knot satisfies the conditions for $T$ in
\lemTangles.
Thus there exists a tangle $T'$, that satisfies the conditions described in
\lemTangles.
Therefore $T'$ can be top closed to create the unlink and by \figComplete, it
can be full closed to create the original knot.
Thus, $T'$ matches the conditions of the tangle described in \thmRibbon.
\end{proof}
\papersection{Symmetric Unions}

Here is a table of all 21 ribbon knots with 10 crossings or fewer in tangle
form.
Each was originally a symmetric union provided by \cite{oneknot} and
\cite{manyknots} and was deformed to appear as the full closure of a tangle with
the extra stitching of $(1_t$, $4_t)$.
\end{paper}

\setlength{\tabcolsep}{12pt}
\fontsize{12pt}{1em}\selectfont
\begin{tabular}{cccc}
\scalebox{0.17}{\svg{diagrams/6_1}}&\scalebox{0.17}{\svg{diagrams/8_8}}&
\scalebox{0.17}{\svg{diagrams/8_9}}&\scalebox{0.17}{\svg{diagrams/8_20}}\\
$6_1$&$8_8$&$8_9$&$8_{20}$\\
&&&\\
\scalebox{0.17}{\svg{diagrams/9_27}}&\scalebox{0.17}{\svg{diagrams/9_41}}&
\scalebox{0.17}{\svg{diagrams/9_46}}&\scalebox{0.17}{\svg{diagrams/10_3}}\\
$9_{27}$&$9_{41}$&$9_{46}$&$10_3$\\
&&&\\
\scalebox{0.17}{\svg{diagrams/10_22}}&\scalebox{0.17}{\svg{diagrams/10_35}}&
\scalebox{0.17}{\svg{diagrams/10_42}}&\scalebox{0.17}{\svg{diagrams/10_48}}\\
$10_{22}$&$10_{35}$&$10_{42}$&$10_{48}$\\
&&&\\
\scalebox{0.17}{\svg{diagrams/10_75}}&\scalebox{0.17}{\svg{diagrams/10_87}}&
\scalebox{0.17}{\svg{diagrams/10_99}}&\scalebox{0.17}{\svg{diagrams/10_123}}\\
$10_{75}$&$10_{87}$&$10_{99}$&$10_{123}$\\
&&&\\
\scalebox{0.17}{\svg{diagrams/10_129}}&\scalebox{0.17}{\svg{diagrams/10_137}}&
\scalebox{0.17}{\svg{diagrams/10_140}}&\scalebox{0.17}{\svg{diagrams/10_153}}\\
$10_{129}$&$10_{137}$&$10_{140}$&$10_{153}$\\
&&&\\
\scalebox{0.17}{\svg{diagrams/10_155}}&&&\\
$10_{155}$&&&
\end{tabular}

\begin{paper}
The locations of the cuts that result in the ends $1_b$ and $2_b$ as well as
$3_b$ and $4_b$ are, in general, not very interesting.
It is merely a location on the knot that can be kept on the outside while
untangling the top closure of the tangle.
However, in the symmetric union, the locations of the two other cuts can always
be obtained by symmetrically deforming the knot until one of the bridges appears
above any crossings on the axis of symmetry.
Then the knot can be cut just above this bridge along the strands that cross it
closest to the axis.

\newsavebox{\mirrorR}
\sbox{\mirrorR}{\reflectbox{\fontsize{9pt}{1em}\selectfont$R$}}

\svg{diagrams/question}
\papercaption{Question}{Here, $R$ represents a tangle whose mirror image is
\usebox{\mirrorR}.
They are connected by $X$, a series of crossings and two bridges.
$U$ is the unlink.
In the lower diagram, the knot has been cut and restitched just outside of the
upper bridge.
The lower strand was pulled under the bridge after being stitched.}

\begin{paperquestion}{Cuts}
For any ribbon knot in its symmetric union presentation, can the location of the
cuts between $(1_t$, $4_t)$ and $(2_t$, $3_t)$ be generalized to be along the
two strands that are first to cross one of the 2 bridges of the symmetric union?
\end{paperquestion}
\papersection{Database}

A database of the planar diagram crossing information of the 21 ribbon
knots has been created.
This notation consists of a list of crossings of the form $X_{a,b,c,d}$ which
labels the four strands in a crossing from the lower incoming strand and
continues counterclockwise.

~~~~~~~~~~~\scalebox{0.5}{\svg{diagrams/crossing}}
\vspace{12pt}
\papercaption{Crossing}{This shows a typical crossing.
The numbering stars from the lower incoming segment and proceeds
counterclockwise.
Thus, this crossing would be labeled $X_{a,b,c,d}$.}

The knots are encoded as they appear in the table above.
The tangle information consists of the list of segments of the knot of each
strand of the tangle before any stitching between crossings is done.
A strand in a crossing will carry the number of the knot segment going into it.
Here is knot $6_1$ as an example.

\svg{diagrams/example}
\papercaption{Example}{This depicts all the segments of knot $6_1$ as numbered.
The numbering always starts from strand 1 and proceeds along the knot.
The planar diagram crossing information of $6_1$ is $X_{2,8,3,7}$
$X_{5,14,6,15}$ $X_{8,20,9,19}$ $X_{3,10,4,11}$ $X_{11,6,12,7}$ $X_{15,4,16,5}$
$X_{17,16,18,17}$ $X_{18,10,19,9}$ $X_{20,2,21,1}$ $X_{21,12,22,13}$
$X_{22,14,1,13}$.
Meanwhile, the tangle information of $6_1$ is \{\{1, 2, 3, 4, 5, 6, 7, 8\},
\{12, 11, 10, 9\}, \{13, 14, 15, 16\}, \{22, 21, 20, 19, 18, 17\}\}.}

The full database may be accessed at TODO URL.
\papersection{References}
\begin{thebibliography}{}
\bibitem{oneknot}
Lamm, Christoph.
\textit{Symmetric union presentations for 2-bridge ribbon knots.}
\texttt{arXiv:math/0602395}
\bibitem{manyknots}
Lamm, Christoph.
\textit{Symmetric unions and ribbon knots.}
Osaka J. Math. 37 (2000), no. 3, 537--550
\end{thebibliography}
\end{paper}
\end{document}
