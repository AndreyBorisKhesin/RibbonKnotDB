\documentclass[twoside]{article}
\usepackage{amsmath}
\usepackage{amssymb}
\usepackage{amsthm}
\usepackage{capt-of}
\usepackage{caption}
\captionsetup{labelformat=empty,labelsep=none}
\usepackage[strict]{changepage}
\usepackage{chngcntr}
\usepackage[americanvoltage, siunitx]{circuitikz}
\usepackage{color,colortbl}
\usepackage{fancyhdr}
\usepackage[T1]{fontenc}
\usepackage{gensymb}
\usepackage[margin=1in]{geometry}
\usepackage{graphicx}
\usepackage{import}
\usepackage{indentfirst}
\usepackage{mathptmx}
\usepackage{mathrsfs}
\usepackage{multicol}
\usepackage{multirow}
\usepackage{pgfplots}
\usepackage{pgfplotstable}
\usepgfplotslibrary{external}
\usepackage{siunitx}
\usepackage{tabu}
\usepackage{tikz}
\usetikzlibrary{positioning,matrix,shapes,chains,arrows}
\tikzexternalize[prefix=precompiled_figures/]
\usepackage{xspace}

% Indent
\setlength{\parindent}{0.3in}

\newcounter{papertheoremamount}
\newcommand\theorems[0]{\newtheorem{conjecture}[subsection]{Conjecture}
\newtheorem{corollary}[subsection]{Corollary}
\newtheorem{lemma}[subsection]{Lemma}\newtheorem{remark}[subsection]{Remark}
\newtheorem{theorem}[subsection]{Theorem}
\newtheorem{question}[subsection]{Question}}
\newcommand\subtheorems[0]{\stepcounter{papertheoremamount}
\newtheorem{conjecture}[subsubsection]{Conjecture}
\newtheorem{corollary}[subsubsection]{Corollary}
\newtheorem{lemma}[subsubsection]{Lemma}
\newtheorem{remark}[subsubsection]{Remark}
\newtheorem{theorem}[subsubsection]{Theorem}
\newtheorem{question}[subsubsection]{Question}}

% Title section
\pagestyle{fancy}
\thispagestyle{empty}
\renewcommand{\headrulewidth}{0pt}
\newcommand\papertitle[1]{{\centering\fontsize{20pt}{1em}\textsc{#1}\\\mbox{}\\}
\fancyhead[OC]{\fontsize{12pt}{12pt}\selectfont\textit{#1}}}
\newcounter{people}
\newcommand\paperauthortext[4]{{\centering\fontsize{12pt}{1em}\selectfont
\textsc{#1. #2}\\[-0.1em]{\fontsize{9pt}{1em}\selectfont\textit{\ifx&#3&
\vspace{-1em}\else#3\fi}}\\\mbox{}\\
\fancyhead[EC]{\fontsize{12pt}{12pt}\selectfont\textit{#4}}}}
\newcommand\paperauthor[3]{{\stepcounter{people}
\ifnum\value{people}=1
{\paperauthortext{#1}{#2}{#3}{#1. #2}
\global\def\auth{#2\xspace}}
\else\ifnum\value{people}=2
{\paperauthortext{#1}{#2}{#3}{\auth and #2}}
\else{\paperauthortext{#1}{#2}{#3}{\auth et al}}\fi\fi}}
\newcommand\paperdate[1]{{\centering\fontsize{9pt}{1em}\selectfont\text{
(Received #1)}\\[2em]}}

% Page header
\newcommand{\paperheader}[1]{\fancyhead[EC]{\fontsize{12pt}{12pt}\selectfont
\textit{#1}}}
\fancyhead[RO, EL]{\fontsize{12pt}{12pt}\selectfont\thepage}
\cfoot{}

\makeatletter
\newenvironment{paperadjustwidth}[2]{
  \begin{list}{}{
    \setlength\partopsep\z@
    \setlength\topsep\z@
    \setlength\listparindent\parindent
    \setlength\parsep\parskip
    \linespread{0}\selectfont
    \@ifmtarg{#1}{\setlength{\leftmargin}{\z@}}
                 {\setlength{\leftmargin}{#1}}
    \@ifmtarg{#2}{\setlength{\rightmargin}{\z@}}
                 {\setlength{\rightmargin}{#2}}
    }
    \item[]}{\end{list}}
\makeatother

% Abstract environment
\newenvironment{paperabstract}
{\begin{paperadjustwidth}{0.5in}{0.5in}\bgroup\fontsize{9pt}{1em}\selectfont
\hspace{0.5in}}
{\egroup\end{paperadjustwidth}}

% Report environment
\setlength\columnsep{0.5in}
\newenvironment{paper}
{\begin{multicols*}{2}\bgroup\fontsize{12pt}{13.8pt}\selectfont}
{\egroup\end{multicols*}}

%Sources
\newsavebox{\sourcebox}
\newcommand{\papersource}[1]{
\vspace{-2em}
\text{}\\*
\fontsize{9pt}{10.35}\selectfont
\noindent\renewcommand{\labelenumi}{}
\savebox{\sourcebox}{\parbox{3in}{\begin{enumerate}
%\vspace{-\baselineskip}
\setlength{\leftmargini}{-1ex}
\setlength{\leftmargin}{-1ex}
\setlength{\labelwidth}{0pt}
\setlength{\labelsep}{0pt}
\setlength{\listparindent}{0pt}
\item\textit{\hspace{-0.35in}#1}
\end{enumerate}}}
\usebox{\sourcebox}
}

%Section headers
\newcounter{papersectioncounter}
\newcounter{papersubsectioncounter}[papersectioncounter]
\newcommand\papersection[1]{\stepcounter{papersectioncounter}
\stepcounter{section}
\begin{center}\Roman{papersectioncounter} \textsc{#1}\end{center}}
\newcommand\papersubsection[1]{\stepcounter{papersubsectioncounter}
\addtocounter{subsection}{\thepapertheoremamount}
\setcounter{subsubsection}{0}
{\begin{center}
\Roman{section}.\Roman{papersubsectioncounter}
\textsc{#1}\\[0.5em]\end{center}}}

%equation
\newcounter{paperequationcounter}
\newcommand\paperequation[3]{{
\stepcounter{paperequationcounter}
\mbox{}\vspace{-0.75em}
\begin{equation*}
#1
\tag*{\fontsize{12pt}{1em}\selectfont
$\begin{array}{r}
\cr{\text{[\arabic{paperequationcounter}]}}
\cr{\fontsize{9pt}{1em}\selectfont\textit{\ifx\\#2\\~\else(\fi#2\ifx\\#2\\~
\else)\fi}}
\end{array}$}
\end{equation*}
}
\expandafter\edef\csname eq#3\endcsname{[\arabic{paperequationcounter}]\noexpand
\xspace}}

% Where
\newcommand{\paperwherevar}[3]{&$#1$ & #2 \ifx\\#3\\~\else($\smash{\text{\si{\fi
#3\ifx\\#3\\~\else}}}$)\fi\\}
\newenvironment{paperwhere}
{\bgroup\fontsize{9pt}{9pt}\selectfont Where:\vspace{2pt}\\\begin{tabular}
{rr@{ = }p{2.42in}}}
{\end{tabular}\egroup\vspace{5pt}}

% Tables
\definecolor{LineGray}{gray}{0.5}
\newtabulinestyle{outer=2.25pt LineGray}
\newtabulinestyle{inner=0.75pt LineGray}
\tabulinesep=1.5pt

\newcommand{\paperiline}[0]{\tabucline[inner]{-}}
\newcommand{\paperoline}[0]{\tabucline[outer]{-}}

% Index column type
\newcolumntype{I}{X[-5,c]}
% Column type with uncertainty
\newcolumntype{U}{@{}X[-5,r]@{$\pm$}X[-5,l]@{}}
% Column type without uncertainty
\newcolumntype{C}{@{}X[-5,c]@{}}

\newcounter{papertableindexcounter}
\newcommand{\papertableindexheader}[0]{\multirow{2}{*}{\textsc{Index}}}
\newcommand{\papertableindex}[0]{\stepcounter{papertableindexcounter}
\arabic{papertableindexcounter}}
\newcommand{\papertableuheadersymbol}[1]{&\multicolumn{2}{c|[inner]}{$#1$}}
\newcommand{\papertableuheadersymbole}[1]{&\multicolumn{2}{c|[outer]}{$#1$}}
\newcommand{\papertableuheaderunit}[1]{&\multicolumn{2}{c|[inner]}{(#1)}}
\newcommand{\papertableuheaderunite}[1]{&\multicolumn{2}{c|[outer]}{(#1)}}
\newcommand{\papertablecheadersymbol}[1]{&$#1$}
\newcommand{\papertablecheaderunit}[2]{&($\pm$#1 #2)}

% Value in table with uncertainty.
\newcommand{\papertableuval}[2]{& #1 & #2}
% Value in table without uncertainty.
\newcommand{\papertablecval}[1]{& #1}

\newenvironment{papertable}[1]
{\setcounter{papertableindexcounter}{0} 
\begin{tabu} to \linewidth {#1}}
{\end{tabu}\vspace{12pt}}

%Figure counter
\newcounter{paperfigurecounter}
\newcommand{\papercaption}[1]{
\vspace{-24pt}
\bgroup
\stepcounter{paperfigurecounter}
\captionof{figure}{\fontsize{9pt}{9pt}\selectfont
\hspace{0.3in}Fig \arabic{paperfigurecounter}. #1}\egroup}

\newcommand{\paperaxis}[9]
{title=#1,
axis x line = bottom,
xmin=#4,xmax=#6,
axis y line = left,
ymin=#5,ymax=#7,
height = 180pt,
grid=both,
x axis line style=-,
y axis line style=-,
x tick label style={
/pgf/number format/.cd,
fixed,
fixed zerofill,
precision=#8,
/tikz/.cd
},
y tick label style={
/pgf/number format/.cd,
fixed,
fixed zerofill,
precision=#9,
/tikz/.cd
}
}
\newcommand{\paperaxisxlabel}[2]{
xlabel=\fontsize{10pt}{1em}\selectfont#1$(#2)\rightarrow$}
\newcommand{\paperaxisylabel}[2]{
ylabel=\fontsize{10pt}{1em}\selectfont#1$(#2)\rightarrow$}
\newcommand{\papergraphoutline}[4]{
\addplot [mark=none,line width=0.75pt] coordinates {
(#1,#2)
(#1,#4)
(#3,#4)
(#3,#2)
(#1,#2)
};}

\newenvironment{papergraph}{
\begin{tikzpicture}
\begin{axis}
}
{\end{axis}
\end{tikzpicture}}

\newcommand{\abs}[1]{\left\lvert#1\right\rvert}
\newcommand{\oo}[0]{\infty}
\newcommand{\sigmaSum}[3]{\sum\limits_{#1}^{#2} #3}
\newcommand{\limto}[3]{\lim\limits_{#1\rightarrow#2}#3}
\renewcommand{\d}[0]{\mathrm{d}}
\newcommand{\cross}[0]{\times}
\newcommand{\lp}{\left(}
\newcommand{\rp}{\right)}
\newcommand\pars[1]{\lp#1\rp}
\newcommand\sqbrack[1]{\left[#1\right]}
\newcommand\R{\mathbb{R}}
\newcommand\di{\partial}
\newcommand\x{\times}
\newcommand\del{\nabla}

\diagrams
\theorems
\begin{document}
\papertitle{A Tabulation of Ribbon Knots in Tangle Form}
%\paperauth{D}{Bar-Natan}{}
\paperauth{A}{Khesin}{University of Toronto}
\begin{paperabs}
It is known that there are 21 ribbon knots with 10 crossings or fewer.
All 21 of these knots can be expressed as symmetric unions (see \cite{many} and
\cite{one}).
We show that for each of these ribbon knots, a tangle with 4 strands can be
found that satisfies the property that a specific closure of that tangle results
in the corresponding knot while another closure results in the 2-component
unlink.
We show that a tangle satisfying similar conditions exists for any ribbon knot
and that the knot corresponding to such a tangle is always a ribbon knot.
We also provide diagrams and the tangle information of the first 21 ribbon
knots.
\end{paperabs}
\begin{paper}
\papersec{Introduction}

There are 250 knots with 10 crossings or fewer.
Of these, 21 satisfy a particular set of conditions to be deemed \textit{ribbon
knots}.
In the Rolfsen Table, these knots have been given the indices $6_1$, $8_8$,
$8_9$, $8_{20}$, $9_{27}$, $9_{41}$, $9_{46}$, $10_3$, $10_{22}$, $10_{35}$,
$10_{42}$, $10_{48}$, $10_{75}$, $10_{87}$, $10_{99}$, $10_{123}$, $10_{129}$,
$10_{137}$, $10_{140}$, $10_{153}$, and $10_{155}$.
It has been shown that these 21 knots can be expressed as \textit{symmetric
unions} (see \cite{many} and \cite{one}).
A symmetric union is a knot that is symmetric on either side of a central axis,
with the exception of crossings located on this axis.
These symmetric unions can be used to construct presentations of these ribbon
knots that are mathematically interesting.

\paperfig{Singularities}{\svgsize{clasp}{0.4\columnwidth}\hfill
\svgsize{ribbon}{0.4\columnwidth}\\

\noindent Clasp Singularity\hfill Ribbon Singularity}
{A clasp singularity and a ribbon singularity.
A ribbon singularity is the image of two curves, one that connects two of the
disc's boundary points and one that connects none.
A clasp singularity occurs when both curves connect one such point.
The translucent regions are the images of the discs, with the gray lines
representing the discs' boundaries, the knots.
The black lines are the singularities.}

A knot is the boundary of a disc embedded in three-dimensional space.
Unless the embedding is trivial and its boundary is the unknot, this disc will
have self-intersections, or \textit{singularities}.
A ribbon knot is a knot that is the boundary of an embedded disc that contains
only ribbon singularities.
A ribbon singularity occurs when each curve of the disk's self-intersections is
the image of a curve connecting two points on the disk's boundary and a curve
connecting two non-boundary points (see \figSingularities).
We note that in such a case, the two strands that pass through the endpoints of
the singularity will be antiparallel.

\papersvg{Presentation}{presentation}{The ribbon presentation of $6_1$.
Note that there are two rectangular components which are the images of two
discs.
The ribbon connecting them is the unzipped image of a strand connecting the
discs.
Since the only intersections in a ribbon presentations are between the ribbon
and the components, any ribbon presentation clearly forms a ribbon knot.}

A \textit{ribbon presentation} of a ribbon knot depicts the knot as the image of
a series of disks, each connected to the next by an arc, the ribbon, which is
unzipped into two antiparallel strands after applying the mapping that sends the
discs to the knot (see \figPresentation).

It is important to note that if we were to fill in the image of disc whose
boundary is a ribbon presentation, then the only singularities will occur when
the ribbon intersections one of the components.
These will clearly be ribbon singularities (see \figSingularities).
Thus, any knot in a ribbon presentation is a ribbon knot, hence the name.

\papersec{Tangles}

A \textit{tangle} is a collection of crossings with a certain number of loose
ends.
An $n$-strand tangle will have $2n$ such ends.

\paperfig{Tangle}{\size{9}{
\hspace{0.19\columnwidth}$1_t$\hspace{0.16\columnwidth}$2_t$
\hspace{0.04\columnwidth}\dots\hspace{0.02\columnwidth}$(n-1)_t$
\hspace{0.105\columnwidth}$n_t$\\
\svgc{tangle}\\
\indent\hspace{0.19\columnwidth}$1_b$\hspace{0.16\columnwidth}$2_b$
\hspace{0.04\columnwidth}\dots\hspace{0.02\columnwidth}$(n-1)_b$
\hspace{0.09\columnwidth}$n_b$}}
{An $n$-strand tangle.
In a tangle, the ends do not have to be drawn along the top and bottom of the
tangle; this is done for convenience.
The $2n$ ends could be located anywhere on the perimeter.
We label the ends of an $n$-strand tangle from $1_t$ to $n_t$ along the top and
$1_b$ to $n_b$ along the bottom.
The central R represents a knot that shows that the dashed strands are tangled,
but link up with their corresponding ends the way they do in a pure braid.}

For convenience, a tangle is often drawn like a pure braid, with $n$ ends along
both the top and the bottom where the order of the strands along the top is the
same as along the bottom.
Thus, if we were to follow any strand of a tangle, we would end right below
where we started.
We denote the tops of the strands by $1_t,~2_t,~\dots,~n_t$, and the
bottoms  by $1_b,~2_b,~\dots,~n_b$ (see \figTangle).
This means that $k_t$ is connected through the tangle to $k_b$, where $k$ is
some value.

A \textit{closure} of a tangle is a way of pairwise stitching the ends of the
tangle together to reduce the resulting number of strands.

\paperfig{Top}{\size{9}{
\hspace{0.1\columnwidth}$1_t$\hspace{0.07\columnwidth}$2_t$
\hspace{0.07\columnwidth}$3_t$\hspace{0.127\columnwidth}\dots
\hspace{0.04\columnwidth}$(2n-2)_t$ $(2n-1)_t$ $(2n)_t$\\
\svgc{top}\\
\indent\hspace{0.1\columnwidth}$1_b$\hspace{0.07\columnwidth}$2_b$
\hspace{0.06\columnwidth}$3_b$\hspace{0.12\columnwidth}\dots
\hspace{0.04\columnwidth}$(2n-2)_b$ $(2n-1)_b$ $(2n)_b$}}
{A top closure of a $2n$-strand tangle.
The top closure leaves every bottom end open and leaves all the upper ends
closed.
After the closure, the tangle has only $n$ strands instead of $2n$.
Only 4 of the $n$ closures are shown here.}

\begin{paperdef}{TopClosure}{For a $2n$-strand tangle, we define the
\textit{top closure} of a tangle to be the closure that stitches together the
pairs $(1_t,~2_t),~(3_t,~4_t),~\dots,~((2n-1)_t,~(2n)_t)$ to create
$n$ strands (see \figTop).}\end{paperdef}

\paperfig{Full}{\size{9}{
\hspace{0.1\columnwidth}$1_t$\hspace{0.07\columnwidth}$2_t$
\hspace{0.07\columnwidth}$3_t$\hspace{0.127\columnwidth}\dots
\hspace{0.04\columnwidth}$(2n-2)_t$ $(2n-1)_t$ $(2n)_t$\\
\svgc{full}\\
\indent\hspace{0.1\columnwidth}$1_b$\hspace{0.07\columnwidth}$2_b$
\hspace{0.06\columnwidth}$3_b$\hspace{0.12\columnwidth}\dots
\hspace{0.04\columnwidth}$(2n-2)_b$ $(2n-1)_b$ $(2n)_b$}}
{A full closure of a $2n$-strand tangle.
It is important to note that the full closure reproduces the top closure along
the bottom ends, and closes the top ends similarly, but offset by 1 strand.
This closure leaves ends $1_t$ and $(2n)_t$ open.
After the closure, the tangle has only 1 strand instead of $2n$.
Only 6 of the $2n-1$ closures are shown here fully.
The break in the central top closure represents the presence of more closed
strands.}

\begin{paperdef}{FullClosure}{For a $2n$-strand tangle, we define the
\textit{full closure} of a tangle to be the closure that stitches together the
pairs $(1_b,~2_b),~(2_t,~3_t),~(3_b,~4_b)$, \dots,
$((2n-2)_t,~(2n-1)_t),~((2n-1)_b,~(2n)_b)$ to create 1 strand (see
\figFull).}\end{paperdef}

If the inside of the tangle is trivial, then in the top closure, the strands
create an untangle which can be turned into the unlink by closing the strands
along the bottom as depicted in the full closure (see \figFull).

These closures allow us to show an important property of ribbon knots.

\begin{paperthm}{Ribbon}
$K$ is a ribbon knot if and only if there exists a $2n$-strand tangle $T$ such
that the top closure of $T$ creates the $n$-strand untangle and the full closure
of $T$ creates a tangle with 1 strand, identical to the knot $K$ when
$(1_t,~(2n)_t)$ is closed.
\end{paperthm}
\begin{proof}
We know that the top closure of $T$ effectively results in $n$ ribbons that can
be completely untangled (see \figTop).
Since the ribbons can be untangled, the only self-intersections that the surface
of the tangle can have are going to be ribbon singularities, as only ribbon
singularities can occur when the untangling process is undone.\\

\papersvg{Proof}{proof}
{The result of top closing a tangle, closing it symmetrically
along the bottom, and then connecting each resulting unlink to the next using a
ribbon.
Connecting the unlinks involves making the connections
$(2_t,~3_t),~(4_t,~5_t),~\dots,~((2n-2)_t,~(2n-1)_t)$ with a pair of strands.
Note that this results in one long strand running along the top of the knot, so
these connections are just the full closure of the tangle along with the
connection $(1_t,~(2n)_t)$.}

Now those $n$ ribbons can be closed along the bottom to create $n$ unknots,
which are the $n$-component unlink.
Then each unknot can be connected to the next with a very thin ribbon by
connecting the pairs $(2_t,~3_t),~(4_t,~5_t),~\dots,~((2n-2)_t,~(2n-1)_t)$ with
a pair of strands each.
These ribbons might pass through the surface of the knot, but since they are
ribbons, this would only add ribbon singularities to the knot.
Since we had top closed the knot earlier, what we have effectively done is
connected each top strand to the next, starting from $1_t$ and ending with
$(2n)_t$ (see \figProof).
The remaining connections make the full closure (see \figFull).
However, connecting each top strand to the next by connecting
$(1_t,~2_t),~(3_t,~4_t),~\dots,~((2n-1)_t,~(2n)_t)$ is the same as connecting
$(1_t,~(2n)_t)$.
This means that we have added the one connection to the full closure to create
the knot $K$ and we have shown that the resulting knot only contains ribbon
singularities.
Therefore, $K$ must be a ribbon knot.

Conversely, consider an $n$-strand tangle $T$.
All $2n$ of its ends can be placed along the top and numbered $1_t$ to $(2n)_t$.

\paperfig{Lemma}{\svgsize{lemmaone}{0.4\columnwidth}\\

\vspace{-4em}\hspace{0.3in}\hspace{14.8ex}\size{20}{$\equiv$}\\

\vspace{-4.3em}\hfill\svgsize{lemmatwo}{0.4\columnwidth}\\

\hspace{0.3in}\hspace{15.1ex}\size{20}{$\Downarrow$}\\

\svgsize{lemmathree}{0.4\columnwidth}\\

\vspace{-4em}\hspace{0.3in}\hspace{14.8ex}\size{20}{$\equiv$}\\

\vspace{-4.3em}\hfill\svgsize{lemmafour}{0.4\columnwidth}\\}
{A visual representation of an implication involving tangles.
The top equality shows that the tangle T can be top closed to
form the unlink, the result of top closing the untangle U.
The implication states that there must exist a tangle T' that can be closed
along the bottom as shown to create T.}

\begin{paperlem}{Tangles}
If an $n$-strand tangle $T$ with all $2n$ of its ends placed along the top can
be top closed to create the $n$-component unlink, then $\exists$ a $2n$-strand
tangle $T'$ s.~t.~applying the bottom part of the full closure results in $T$
(see \figLemma).
\end{paperlem}

\begin{proof}[Proof of Lemma]
We know that after top closing $T$, a continuous spatial deformation can be
applied to turn it into the unlink.
Once we have untangled $T$, a strand can be attached to each of the resulting
unknots.
This strand can be extended to below the boundary of the tangle and fix it to
some arbitrary point.
If we were to reverse the earlier spatial deformation, the parts of the strands
that we added that were outside of the tangle would remain there, while the
remainder would be tangled up inside $T$.
These strands can now be unzipped into ribbons, and think of them as part of the
tangle, resulting in a new path for each strand of $T$.
Thus, these new paths follow their old paths, go on a long detour until they
reach a point outside of the tangle, turn around and come back, and then resume
their old path.

We see that by aligning the loops that these paths trace along the bottom,
they can be viewed as closures to a series of strands.
Thus, with these paths the tangle is $T$, but without these paths it is $T'$,
as when this $T'$ is closed along the bottom by these loops, we get long ribbons
that extend from the strands of $T$, but are nevertheless equivalent to $T$.
Therefore, such a tangle $T'$ exists.
\end{proof}

\papersvg{Lowered}{lowered}
{A representation of a knot in ribbon presentation after all of the
intersections were enclosed in one large tangle labeled R.
The components of the knots are depicted exactly as they are while the
intersections of the ribbons and the components are not shown.}

For any knot in its ribbon presentation, all of the lower halves of the
components can be enclosed in one large tangle (see \figPresentation).
Next, all of the ribbons can be moved so that all of the connections between
ribbons and components occur outside of this tangle but any intersections occur
inside (see \figLowered).\\

\papersvg{Twisted}{twisted}
{A representation of a knot in ribbon presentation after all of the
intersections were enclosed in one large tangle labeled R and all of the ribbons
were extracted out of it.
The ribbons of the knots are depicted exactly as they are while the
intersections of the components are not shown.}

After we have enclosed the knot in a tangle, all the ribbons can be pulled out
of this tangle.
However, the tangle might be non-trivial since the ribbons could intersect the
components.
Fortunately, all of these intersections form ribbon singularities, so they can
be moved them back into the tangle.

If a ribbon passed through a component before we pulled it up, there will be a
strand on either side of the ribbon that can be moved back into the tangle as
the area below and above the ribbons is freely connected to the tangle (see
\figTwisted).

However, since the top part of the knot no longer has any intersections, it is
equivalent to the top half of the full closure of the large tangle, as well as
the connection $(1_t,~(2n)_t)$ (see \figFull).

We know that since every ribbon knot has a ribbon presentation, then removing
the ribbons that connect the components of the knot results in the $n$-component
unlink (see \figPresentation).
We also know that if were to take our constructed tangle that encloses all of
the crossings of our knot and top close the tangle, the result would be very
similar the full closure that we constructed earlier (see \figTop).
The reason for this is that the top closure would close each component with its
other end, effectively removing all of the connecting ribbons (see \figTwisted).

Thus, the top closure of our tangle forms the unlink, as it removes all the
ribbons from a ribbon knot.
This means that this tangle satisfies the conditions of $T$ in \lemTangles.
As a consequence, we know that there must exist a tangle $T'$ that can be closed
along the bottom to result in $T$.

We know that this $T'$ can be top closed to result in the untangle, as $T$ can
be top closed to result in the unlink (see \figLemma).
Additionally, if the bottom half of the full closure of $T'$ results in $T$ and
the top half of the full closure with the connection $(1_t,~(2n)_t)$ of $T$
results in our knot, then the full closure of $T'$ with the same connection also
results in our knot.

Therefore, this tangle given to us by \lemTangles satisfies the conditions of
the theorem and we have shown its existence, proving the theorem.
\end{proof}

\papersec{Symmetric Unions}

It is known that all 21 ribbon knots with 10 crossings or fewer can be drawn as
symmetric unions (see \cite{many} and \cite{one}).
This allowed us to find a tangle satisfying the conditions of \thmRibbon for
each of these 21 ribbon knots.
This was possible due to the fact that adjusting a few connections in a
symmetric union makes it untwist and untangle as we would like it to after
applying the top closure (see \figTop).
Surprisingly, each of these tangles only had 2 components, despite the fact that
there is no requirement stipulating that this must be the case.

\begin{paperqtn}{Components}
Do all the tangles for ribbon knots that are generated by \thmRibbon have 2
components?
\end{paperqtn}

These tangles are obtained from the original knots by locating 4 places in which
to cut the knot.
The result is the tangle and if it is labeled properly from $1_t$ to $4_b$ then
the full closure will undo the earlier cuts and the top closure will untangle it.

The locations of the two cuts that result in the ends $1_b$ and $2_b$, as well
as $3_b$ and $4_b$, are, in general, not very interesting.
They are merely a location on the knot that can be kept on the outside while
untangling the top closure of the tangle.

However, the two cuts that resulted in the top ends proved to be more
interesting.
We know that in a symmetric union, the central axis contains crossings and at
least two bridges.
One of these bridges can be moved up until it is the uppermost component of the
central axis.\newsavebox{\knotR}\sbox{\knotR}{\reflectbox{\size{9}{R}}}

\paperfig{Question}{\svgsize{questionone}{0.4\columnwidth}\\

\vspace{-6em}\hspace{0.3in}\hspace{14.8ex}\size{20}{$\equiv$}\\

\vspace{-6em}\hfill\svgsize{questiontwo}{0.4\columnwidth}\\

\hspace{0.3in}\hspace{15.1ex}\size{20}{$\Downarrow$}\\

\svgsize{questionthree}{0.4\columnwidth}\\

\vspace{-6em}\hspace{0.3in}\hspace{14.8ex}\size{20}{$\equiv$}\\

\vspace{-6.7em}\hfill\svgsize{questionfour}{0.4\columnwidth}}
{A visual representation of an implication involving tangles.
This shows that if a knot is a symmetric union represented by a tangle R, its
reflection \usebox{\knotR}, a series of crossings and bridges X, and the
appropriate connections, then cutting in and reconnecting the strands in a
certain manner results in the closed untangle U.
In the resulting diagram, the knot has been cut just outside of the upper bridge
and the ends were connected to their reflections.
The inner of these connections was pulled under the original bridge after the
connections are made.}

After moving the bridge, we can follow the bridge both ways until we reach the
first crossing it makes with the rest of the knot, and move along the strands
that the bridge crosses towards the side of the bridge that faces up.
There, we always find ourselves in a pair of suitable locations to make the two
cuts that form the four upper ends of the resulting tangle for any of the
relevant 21 ribbon knots.

\begin{paperqtn}{Cuts}
For any ribbon knot in its symmetric union presentation, can the location of the
cuts between $(1_t,~4_t)$ and $(2_t,~3_t)$ be generalized to be along the
two strands that are first to cross a bridge of the symmetric union?
\end{paperqtn}

Although it is known that all symmetric presentations are ribbon knots, it is
unknown whether every ribbon knot has a symmetric union presentation.

\papersec{Database}

By applying the process described by \thmRibbon to each of the 21 ribbon knots,
we were able to find the corresponding tangles.
We can write down the tangle information of each of these by enumerating all of
the strands from a common starting point.

We decided to use $1_t$ proceeding downwards.
Thus, the strand descending from $1_t$ is labeled 1, and every time it passes
through a crossing, the value increments.
We can then write down the planar diagram information for the knot.

\paperfig{Crossing}
{\hspace{0.3in}\hspace{0.175\columnwidth}$k$\hspace{0.4\columnwidth}$j$
\begin{center}\svgsize{crossing}{0.4\columnwidth}\end{center}

\hspace{0.3in}\hspace{0.175\columnwidth}$l$\hspace{0.4\columnwidth}$i$\\

\hspace{0.3in}\hspace{0.31\columnwidth}$X_{i,~j,~k,~l}$}
{A right-handed crossing labeled in planar diagram notation.
The lower incoming strand is labeled $i$ and then the remaining three are
labeled $j$, $k$, and $l$, proceeding counterclockwise from $i$.
The crossing is labeled as $X_{i,~j,~k,~l}$.}

Each crossing is labeled with $X_{i,~j,~k,~l}$ where $i$ is the index of the
lower incoming strand and the remaining indices proceed counterclockwise (see
\figCrossing).
The flow of the tangle is down the first strand, up the second, down the third,
and then up the fourth.

\papersvg{Example}{example}
{This depicts all the segments of knot $6_1$ as numbered.
The numbering starts from $1_t$ and proceeds downwards.
The tangle values of $6_1$ are (9, 13, 17) the values of the strands along the
three connections from strands 1 to 4.}

We also need to know where the strands of the tangle have been cut, so we need
to know the three numbers given to the connections $(1_b,~2_b),~(2_t,~3_t)$, and
$(3_b,~4_b)$ in that order.
We do not need to value for $(1_t,~4_t)$ as it is always 1.
For the knot $6_1$, the tangle values are (9, 13, 17) (see \figExample).

We present these tangles for all 21 ribbon knots with 10 crossings or fewer,
drawn with the full closure and the connection $(1_t,~4_t)$ added, effectively
recreating the original knots.
We also present the planar diagram notation and tangle values of each of the
knots as they appear in the table (see below).
All of the data in the second table is also available online at\\
\url{
https://raw.githubusercontent.com/AndreyBorisKhesin/RibbonKnotDB/master/db.csv}.
TODO SHORTEN URL

\papersec{References}

\begin{thebibliography}{}
\bibitem{one}
Lamm, Christoph.
\textit{Symmetric union presentations for 2-bridge ribbon knots.}
\texttt{arXiv:math/0602395}
\bibitem{many}
Lamm, Christoph.
\textit{Symmetric unions and ribbon knots.}
Osaka J. Math. 37 (2000), no. 3, 537--550
\end{thebibliography}

\papersec{Acknowledgements}

\end{paper}

\setlength{\tabcolsep}{12pt}
\begin{tabular}{cccc}
\svgsize{6_1}{0.17\columnwidth}&\svgsize{8_8}{0.17\columnwidth}&
\svgsize{8_9}{0.17\columnwidth}&\svgsize{8_20}{0.17\columnwidth}\\
$6_1$&$8_8$&$8_9$&$8_{20}$\\
&&&\\
\svgsize{9_27}{0.17\columnwidth}&\svgsize{9_41}{0.17\columnwidth}&
\svgsize{9_46}{0.17\columnwidth}&\svgsize{10_3}{0.17\columnwidth}\\
$9_{27}$&$9_{41}$&$9_{46}$&$10_3$\\
&&&\\
\svgsize{10_22}{0.17\columnwidth}&\svgsize{10_35}{0.17\columnwidth}&
\svgsize{10_42}{0.17\columnwidth}&\svgsize{10_48}{0.17\columnwidth}\\
$10_{22}$&$10_{35}$&$10_{42}$&$10_{48}$\\
&&&\\
\svgsize{10_75}{0.17\columnwidth}&\svgsize{10_87}{0.17\columnwidth}&
\svgsize{10_99}{0.17\columnwidth}&\svgsize{10_123}{0.17\columnwidth}\\
$10_{75}$&$10_{87}$&$10_{99}$&$10_{123}$\\
&&&\\
\svgsize{10_129}{0.17\columnwidth}&\svgsize{10_137}{0.17\columnwidth}&
\svgsize{10_140}{0.17\columnwidth}&\svgsize{10_153}{0.17\columnwidth}\\
$10_{129}$&$10_{137}$&$10_{140}$&$10_{153}$\\
&&&\\
\svgsize{10_155}{0.17\columnwidth}&&&\\
$10_{155}$&&&
\end{tabular}

\pagebreak

\begin{table}[h]
\begin{tabularx}{\textwidth}{|c|X|c|}\hline
\brokensize{9}{Knot}&\brokensize{9}{Planar Diagram Notation}&
\brokensize{9}{Tangle Values}\\\hline
\brokensize{9}{$6_1$}&\brokensize{9}{$X_{2,~8,~3,~7}$ $X_{3,~10,~4,~11}$
$X_{5,~14,~6,~15}$ $X_{8,~20,~9,~19}$ $X_{11,~6,~12,~7}$ $X_{15,~4,~16,~5}$
$X_{17,~16,~18,~17}$ $X_{18,~10,~19,~9}$ $X_{20,~2,~21,~1}$ $X_{21,~12,~22,~13}$
$X_{22,~14,~1,~13}$}&\brokensize{9}{(9,~13,~17)}\\\hline
\brokensize{9}{$8_8$}&\brokensize{9}{$X_{2,~10,~3,~9}$ $X_{4,~24,~5,~23}$
$X_{6,~12,~7,~11}$ $X_{7,~14,~8,~15}$ $X_{10,~4,~11,~3}$ $X_{13,~18,~14,~19}$
$X_{15,~8,~16,~9}$ $X_{19,~12,~20,~13}$ $X_{21,~20,~22,~21}$ $X_{22,~6,~23,~5}$
$X_{24,~2,~25,~1}$ $X_{25,~16,~26,~17}$
$X_{26,~18,~1,~17}$}&\brokensize{9}{(5,~17,~21)}\\\hline
\brokensize{9}{$8_9$}&\brokensize{9}{$X_{2,~13,~3,~14}$ $X_{4,~18,~5,~17}$
$X_{9,~8,~10,~9}$ $X_{11,~30,~12,~31}$ $X_{16,~27,~17,~28}$ $X_{18,~30,~19,~29}$
$X_{19,~23,~20,~22}$ $X_{20,~8,~21,~7}$ $X_{21,~6,~22,~7}$ $X_{23,~11,~24,~10}$
$X_{25,~12,~26,~13}$ $X_{26,~4,~27,~3}$ $X_{28,~5,~29,~6}$ $X_{31,~25,~32,~24}$
$X_{32,~2,~33,~1}$ $X_{33,~14,~34,~15}$
$X_{34,~16,~1,~15}$}&\brokensize{9}{(9,~15,~21)}\\\hline
\brokensize{9}{$8_{20}$}&\brokensize{9}{$X_{2,~9,~3,~10}$ $X_{3,~14,~4,~15}$
$X_{7,~6,~8,~7}$ $X_{8,~23,~9,~24}$ $X_{11,~19,~12,~18}$ $X_{13,~4,~14,~5}$
$X_{15,~11,~16,~10}$ $X_{19,~13,~20,~12}$ $X_{21,~20,~22,~21}$
$X_{22,~5,~23,~6}$ $X_{24,~2,~25,~1}$ $X_{25,~16,~26,~17}$
$X_{26,~18,~1,~17}$}&\brokensize{9}{(7,~17,~21)}\\\hline
\brokensize{9}{$9_{27}$}&\brokensize{9}{$X_{3,~27,~4,~26}$ $X_{6,~23,~7,~24}$
$X_{8,~16,~9,~15}$ $X_{9,~2,~10,~3}$ $X_{11,~23,~12,~22}$ $X_{12,~6,~13,~5}$
$X_{14,~19,~15,~20}$ $X_{18,~8,~19,~7}$ $X_{21,~4,~22,~5}$ $X_{24,~14,~25,~13}$
$X_{25,~20,~26,~21}$ $X_{27,~10,~28,~11}$ $X_{28,~2,~29,~1}$
$X_{29,~16,~30,~17}$ $X_{30,~18,~1,~17}$}&\brokensize{9}{(11,~17,~23)}\\\hline
\brokensize{9}{$9_{41}$}&\brokensize{9}{$X_{3,~18,~4,~19}$ $X_{4,~24,~5,~23}$
$X_{7,~14,~8,~15}$ $X_{8,~27,~9,~28}$ $X_{11,~21,~12,~20}$ $X_{12,~1,~13,~2}$
$X_{13,~27,~14,~26}$ $X_{15,~6,~16,~7}$ $X_{17,~30,~18,~31}$ $X_{22,~10,~23,~9}$
$X_{24,~30,~25,~29}$ $X_{25,~17,~26,~16}$ $X_{28,~5,~29,~6}$ $X_{31,~3,~32,~2}$
$X_{32,~19,~33,~20}$ $X_{33,~10,~34,~11}$
$X_{34,~22,~1,~21}$}&\brokensize{9}{(13,~21,~27)}\\\hline
\brokensize{9}{$9_{46}$}&\brokensize{9}{$X_{2,~10,~3,~9}$ $X_{3,~14,~4,~15}$
$X_{5,~12,~6,~13}$ $X_{7,~18,~8,~19}$ $X_{10,~24,~11,~23}$ $X_{13,~4,~14,~5}$
$X_{15,~8,~16,~9}$ $X_{19,~6,~20,~7}$ $X_{21,~20,~22,~21}$ $X_{22,~12,~23,~11}$
$X_{24,~2,~25,~1}$ $X_{25,~16,~26,~17}$
$X_{26,~18,~1,~17}$}&\brokensize{9}{(11,~17,~21)}\\\hline
\brokensize{9}{$10_3$}&\brokensize{9}{$X_{2,~13,~3,~14}$ $X_{3,~18,~4,~19}$
$X_{5,~16,~6,~17}$ $X_{8,~10,~9,~9}$ $X_{10,~24,~11,~23}$ $X_{12,~22,~13,~21}$
$X_{15,~6,~16,~7}$ $X_{17,~4,~18,~5}$ $X_{19,~15,~20,~14}$ $X_{20,~1,~21,~2}$
$X_{22,~12,~23,~11}$ $X_{24,~8,~1,~7}$}&\brokensize{9}{(9,~15,~1)}\\\hline
\brokensize{9}{$10_{22}$}&\brokensize{9}{$X_{3,~17,~4,~16}$ $X_{8,~23,~9,~24}$
$X_{12,~21,~13,~22}$ $X_{14,~8,~15,~7}$ $X_{17,~31,~18,~30}$ $X_{19,~3,~20,~2}$
$X_{20,~9,~21,~10}$ $X_{22,~13,~23,~14}$ $X_{24,~16,~25,~15}$ $X_{25,~6,~26,~7}$
$X_{27,~26,~28,~27}$ $X_{28,~6,~29,~5}$ $X_{29,~4,~30,~5}$ $X_{31,~19,~32,~18}$
$X_{32,~2,~33,~1}$ $X_{33,~10,~34,~11}$
$X_{34,~12,~1,~11}$}&\brokensize{9}{(5,~11,~27)}\\\hline
\brokensize{9}{$10_{35}$}&\brokensize{9}{$X_{3,~12,~4,~13}$ $X_{4,~32,~5,~31}$
$X_{6,~30,~7,~29}$ $X_{8,~14,~9,~13}$ $X_{10,~20,~11,~19}$ $X_{11,~2,~12,~3}$
$X_{15,~24,~16,~25}$ $X_{17,~22,~18,~23}$ $X_{18,~10,~19,~9}$
$X_{23,~16,~24,~17}$ $X_{25,~14,~26,~15}$ $X_{27,~26,~28,~27}$
$X_{28,~8,~29,~7}$ $X_{30,~6,~31,~5}$ $X_{32,~2,~33,~1}$ $X_{33,~20,~34,~21}$
$X_{34,~22,~1,~21}$}&\brokensize{9}{(7,~21,~27)}\\\hline
\brokensize{9}{$10_{42}$}&\brokensize{9}{$X_{3,~39,~4,~38}$ $X_{5,~14,~6,~15}$
$X_{6,~35,~7,~36}$ $X_{9,~13,~10,~12}$ $X_{13,~25,~14,~24}$ $X_{16,~21,~17,~22}$
$X_{20,~33,~21,~34}$ $X_{22,~38,~23,~37}$ $X_{23,~4,~24,~5}$ $X_{25,~9,~26,~8}$
$X_{27,~26,~28,~27}$ $X_{28,~12,~29,~11}$ $X_{29,~10,~30,~11}$
$X_{31,~3,~32,~2}$ $X_{32,~17,~33,~18}$ $X_{34,~7,~35,~8}$ $X_{36,~16,~37,~15}$
$X_{39,~31,~40,~30}$ $X_{40,~2,~41,~1}$ $X_{41,~18,~42,~19}$
$X_{42,~20,~1,~19}$}&\brokensize{9}{(11,~19,~27)}\\\hline
\brokensize{9}{$10_{48}$}&\brokensize{9}{$X_{4,~14,~5,~13}$ $X_{5,~27,~6,~26}$
$X_{7,~29,~8,~28}$ $X_{9,~14,~10,~15}$ $X_{11,~22,~12,~23}$ $X_{12,~4,~13,~3}$
$X_{17,~16,~18,~17}$ $X_{18,~33,~19,~34}$ $X_{20,~35,~21,~36}$
$X_{21,~10,~22,~11}$ $X_{23,~3,~24,~2}$ $X_{27,~7,~28,~6}$ $X_{29,~9,~30,~8}$
$X_{31,~30,~32,~31}$ $X_{32,~15,~33,~16}$ $X_{34,~19,~35,~20}$
$X_{36,~2,~37,~1}$ $X_{37,~24,~38,~25}$
$X_{38,~26,~1,~25}$}&\brokensize{9}{(17,~25,~31)}\\\hline
\brokensize{9}{$10_{75}$}&\brokensize{9}{$X_{4,~14,~5,~13}$ $X_{5,~27,~6,~26}$
$X_{7,~29,~8,~28}$ $X_{9,~14,~10,~15}$ $X_{11,~22,~12,~23}$ $X_{12,~4,~13,~3}$
$X_{17,~16,~18,~17}$ $X_{18,~33,~19,~34}$ $X_{20,~35,~21,~36}$
$X_{21,~10,~22,~11}$ $X_{23,~3,~24,~2}$ $X_{27,~7,~28,~6}$ $X_{29,~9,~30,~8}$
$X_{31,~30,~32,~31}$ $X_{32,~15,~33,~16}$ $X_{34,~19,~35,~20}$
$X_{36,~2,~37,~1}$ $X_{37,~24,~38,~25}$
$X_{38,~26,~1,~25}$}&\brokensize{9}{(9,~17,~25)}\\\hline
\brokensize{9}{$10_{87}$}&\brokensize{9}{$X_{4,~22,~5,~21}$ $X_{6,~14,~7,~13}$
$X_{7,~24,~8,~25}$ $X_{12,~36,~13,~35}$ $X_{15,~38,~16,~39}$ $X_{16,~4,~17,~3}$
$X_{20,~33,~21,~34}$ $X_{23,~15,~24,~14}$ $X_{25,~10,~26,~11}$
$X_{27,~26,~28,~27}$ $X_{28,~10,~29,~9}$ $X_{29,~8,~30,~9}$ $X_{31,~3,~32,~2}$
$X_{32,~17,~33,~18}$ $X_{34,~12,~35,~11}$ $X_{36,~5,~37,~6}$
$X_{37,~22,~38,~23}$ $X_{39,~31,~40,~30}$ $X_{40,~2,~41,~1}$
$X_{41,~18,~42,~19}$ $X_{42,~20,~1,~19}$}&\brokensize{9}{(9,~19,~27)}\\\hline
\brokensize{9}{$10_{99}$}&\brokensize{9}{$X_{2,~29,~3,~30}$ $X_{4,~13,~5,~14}$
$X_{7,~36,~8,~37}$ $X_{11,~21,~12,~20}$ $X_{12,~3,~13,~4}$ $X_{14,~9,~15,~10}$
$X_{16,~26,~17,~25}$ $X_{19,~11,~20,~10}$ $X_{21,~31,~22,~30}$
$X_{24,~18,~25,~17}$ $X_{27,~9,~28,~8}$ $X_{28,~5,~29,~6}$ $X_{31,~19,~32,~18}$
$X_{32,~15,~33,~16}$ $X_{33,~27,~34,~26}$ $X_{35,~34,~36,~35}$
$X_{37,~6,~38,~7}$ $X_{38,~2,~39,~1}$ $X_{39,~22,~40,~23}$
$X_{40,~24,~1,~23}$}&\brokensize{9}{(7,~23,~35)}\\\hline
\brokensize{9}{$10_{123}$}&\brokensize{9}{$X_{2,~23,~3,~24}$ $X_{5,~14,~6,~15}$
$X_{7,~17,~8,~16}$ $X_{8,~3,~9,~4}$ $X_{10,~34,~11,~33}$ $X_{15,~26,~16,~27}$
$X_{17,~25,~18,~24}$ $X_{20,~14,~21,~13}$ $X_{22,~9,~23,~10}$ $X_{25,~7,~26,~6}$
$X_{27,~4,~28,~5}$ $X_{28,~22,~29,~21}$ $X_{29,~12,~30,~13}$
$X_{31,~30,~32,~31}$ $X_{32,~12,~33,~11}$ $X_{34,~2,~35,~1}$
$X_{35,~18,~36,~19}$ $X_{36,~20,~1,~19}$}&\brokensize{9}{(11,~19,~31)}\\\hline
\brokensize{9}{$10_{129}$}&\brokensize{9}{$X_{2,~19,~3,~20}$ $X_{4,~12,~5,~11}$
$X_{6,~28,~7,~27}$ $X_{8,~14,~9,~13}$ $X_{9,~16,~10,~17}$ $X_{12,~6,~13,~5}$
$X_{15,~22,~16,~23}$ $X_{17,~10,~18,~11}$ $X_{18,~3,~19,~4}$
$X_{23,~14,~24,~15}$ $X_{25,~24,~26,~25}$ $X_{26,~8,~27,~7}$ $X_{28,~2,~29,~1}$
$X_{29,~20,~30,~21}$ $X_{30,~22,~1,~21}$}&\brokensize{9}{(7,~21,~25)}\\\hline
\brokensize{9}{$10_{137}$}&\brokensize{9}{$X_{3,~27,~4,~26}$ $X_{8,~13,~9,~14}$
$X_{12,~18,~13,~17}$ $X_{15,~24,~16,~25}$ $X_{16,~7,~17,~8}$ $X_{18,~10,~19,~9}$
$X_{19,~2,~20,~3}$ $X_{21,~5,~22,~4}$ $X_{22,~5,~23,~6}$ $X_{23,~7,~24,~6}$
$X_{25,~14,~26,~15}$ $X_{27,~20,~28,~21}$ $X_{28,~2,~29,~1}$
$X_{29,~10,~30,~11}$ $X_{30,~12,~1,~11}$}&\brokensize{9}{(5,~11,~23)}\\\hline
\brokensize{9}{$10_{140}$}&\brokensize{9}{$X_{2,~11,~3,~12}$ $X_{3,~18,~4,~19}$
$X_{5,~16,~6,~17}$ $X_{9,~8,~10,~9}$ $X_{10,~27,~11,~28}$ $X_{13,~23,~14,~22}$
$X_{15,~6,~16,~7}$ $X_{17,~4,~18,~5}$ $X_{19,~13,~20,~12}$ $X_{23,~15,~24,~14}$
$X_{25,~24,~26,~25}$ $X_{26,~7,~27,~8}$ $X_{28,~2,~29,~1}$ $X_{29,~20,~30,~21}$
$X_{30,~22,~1,~21}$}&\brokensize{9}{(9,~21,~25)}\\\hline
\brokensize{9}{$10_{153}$}&\brokensize{9}{$X_{3,~27,~4,~26}$ $X_{4,~22,~5,~21}$
$X_{7,~6,~8,~7}$ $X_{9,~3,~10,~2}$ $X_{10,~15,~11,~16}$ $X_{13,~24,~14,~25}$
$X_{14,~19,~15,~20}$ $X_{18,~11,~19,~12}$ $X_{22,~6,~23,~5}$
$X_{23,~12,~24,~13}$ $X_{25,~20,~26,~21}$ $X_{27,~9,~28,~8}$ $X_{28,~2,~29,~1}$
$X_{29,~16,~30,~17}$ $X_{30,~18,~1,~17}$}&\brokensize{9}{(7,~17,~23)}\\\hline
\brokensize{9}{$10_{155}$}&\brokensize{9}{$X_{4,~18,~5,~17}$ $X_{9,~8,~10,~9}$
$X_{11,~30,~12,~31}$ $X_{14,~2,~15,~1}$ $X_{16,~27,~17,~28}$
$X_{18,~30,~19,~29}$ $X_{20,~8,~21,~7}$ $X_{21,~6,~22,~7}$ $X_{22,~19,~23,~20}$
$X_{23,~11,~24,~10}$ $X_{25,~12,~26,~13}$ $X_{26,~4,~27,~3}$ $X_{28,~5,~29,~6}$
$X_{31,~25,~32,~24}$ $X_{32,~13,~33,~14}$ $X_{33,~3,~34,~2}$
$X_{34,~16,~1,~15}$}&\brokensize{9}{(9,~15,~21)}\\\hline
\end{tabularx}
\end{table}
\end{document}
